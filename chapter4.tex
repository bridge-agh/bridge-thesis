\chapter{\ChapterTitleWorkOrganization}
\label{sec:organizacja-pracy}

\section{Historia pomysłu projektu}

%%% jak było

\section{Charakterystyka i sposób realizacji projektu}

%%% bardzo ogólna wizja co chcemy mieć
%%% jak podzieliliśmy głowne częsci całej aplikacji
%%% core, backend, front, game-engine, thesis

\section{Weryfikacja postępów w trakcie realizacji}

\subsection{Milestone'y}
%%% najpierw base, pozniej lobby, pozniej gra a na koncu gra z ai
%%% scisle powiazane z epicami

\subsection{Spotkania z promotorem}
%%% to opisujmey w obu przypadkach \subsubsection{Microsoft Teams}

\subsection{Pracownia projektowa}
%%% to opisujmey w obu przypadkach \subsubsection{Microsoft Teams}


\section{Zastosowane praktyki i narzędzia organizacji pracy}

\subsection{Planowanie i projektowanie}

\subsubsection{Shortcut}

\subsubsection{Figma}


\section{Wykorzystywane technologie do realizacji projektu}

\subsection{Komunikacja wewnętrzna}

\subsubsection{Discord}


\subsection{Tworzenie dokumentacji}

\subsection{VSCode}

\subsection{Github Actions}
%%% pluginy itp jak to dziala (auto deploy na githuba release'y)


\subsection{Tworzenie oprogramowania}

\subsubsection{VSCode}

\subsubsection{Produkty Jetbrains}

\subsubsection{Postman}


\subsection{Wdrażanie i testowanie}

\subsubsection{Github Actions (deploy \& tests)}

%%%\subsubsubsection{Konteneryzacja (repozytorium)}

\subsubsection{Oracle Cloud (backend)}

\subsubsection{Vercel (front)}
