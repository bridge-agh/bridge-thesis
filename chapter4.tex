\chapter{\ChapterTitleWorkOrganization}
\label{sec:organizacja-pracy}

\section{Historia pomysłu projektu}

%%% jak było

\section{Charakterystyka i sposób realizacji projektu}

%%% bardzo ogólna wizja co chcemy mieć
%%% jak podzieliliśmy głowne częsci całej aplikacji
%%% core, backend, front, game-engine, thesis

\section{Weryfikacja postępów w trakcie realizacji}

\subsection{Milestone'y}
%%% najpierw base, pozniej lobby, pozniej gra a na koncu gra z ai
%%% scisle powiazane z epicami

\subsection{Spotkania z promotorem}
%%% to opisujmey w obu przypadkach \subsubsection{Microsoft Teams}

\subsection{Pracownia projektowa}
%%% to opisujmey w obu przypadkach \subsubsection{Microsoft Teams}


\section{Zastosowane praktyki i narzędzia organizacji pracy}

\subsection{Planowanie i projektowanie}

\subsubsection{Shortcut}

\subsubsection{Figma}


\section{Wykorzystywane technologie do realizacji projektu}

\subsection{Komunikacja wewnętrzna}

W trakcie tworzenia projektu potrzebna była szybka i efektywna
wymiana wiadomości oraz możliwość współpracy w czasie rzeczywistym.
Wykorzystana została w tym celu platforma Discord~\cite{Discord},
która oferuje rozmowy głosowe, kanały tekstowe i udostępnianie
obrazu na żywo.


\subsection{System kontroli wersji}

Aby synchronizować postępy pracy i ich historię zastosowano
system kontroli wersji Git \cite{Git}. Wraz z nim, do przechowywania
repozytoriów z kodem aplikacji, jak i dokumentacji wykorzystano
platformę GitHub \cite{Github}.

\subsection{Tworzenie dokumentacji}

Do napisania dokumentacji aplikacji użyto języka \LaTeX~\cite{Latex}.
Przesyłając tekst pracy na platformę GitHub, skorzystano z~technologii
GitHub Actions \cite{GithubActions}, którą wykorzystano w celu
generacji dokumentu od razu po opublikowaniu zmian w~tekście pracy.
Umożliwiło to na dostęp do najnowszej wersji dokumentu dla opiekuna
pracy i~prowadzącego pracownię projektową, dzięki czemu była możliwa
ciągła weryfikacja postępów nad pracą.


\subsection{Tworzenie oprogramowania}

Rozwój oprogramowania stanowił kluczowy i niezbędny element w realizacji założeń projektowych. 
Naszym celem było stworzenie funkcjonalnej aplikacji, która integruje odmienne komponenty, wykorzystujące różne technologie. 
Aby sprostać złożonym wymaganiom projektu, skorzystaliśmy z szerokiego spektrum narzędzi.

\subsubsection{VSCode}

Visual Studio Code to zaawansowany edytor kodu stworzony przez Microsoft, który stał się jednym z najpopularniejszych narzędzi wśród programistów.
VS Code wspiera wiele języków programowania i znaczników od JavaScript, TypeScript, Python, PHP, C++ po HTML, CSS, JSON i wiele innych.
Do tego oferuje zaawansowane funkcje autouzupełniania kodu, informacje o typach, podpowiedzi dotyczące parametrów funkcji, info o definicjach i dokumentację.
Dużą zaletą VSCode jest również możliwość personalizacji. Dostępne są tysiące rozszerzeń, które pozwalają na dostosowanie edytora do własnych potrzeb.
Użycie tego edytora znacznie usprawniło rozwój frontendu aplikacji.

\subsubsection{Produkty Jetbrains}

Produkty Jetbrains, w tym dedykowane środowiska programistyczne (IDE) dla poszczególnych języków i ekosystemów, takie jak Java, Python czy PHP, są cenione za zaawansowane funkcje autouzupełniania i dogłębnej analizy kodu. Dostarczają one również potężnych narzędzi do refaktoryzacji, co znacząco ułatwia zarządzanie i optymalizację projektu. PyCharm, specjalizujący się w Pythonie, był pomocny podczas tworzenia modułu logiki gry w brydża, a także w początkowej fazie konstruowania serwera z użyciem FastApi. Z kolei IntelliJ IDEA, ze wsparciem dla języka Scala, znalazł zastosowanie w procesie rozwijania ostatecznej wersji architektury serwerowej. Wykorzystanie tych narzędzi znacząco przyczyniło się do efektywności naszego procesu deweloperskiego.

\subsubsection{Postman}


\subsection{Wdrażanie i testowanie}

\subsubsection{Github Actions (deploy \& tests)}

%%%\subsubsubsection{Konteneryzacja (repozytorium)}

\subsubsection{Oracle Cloud (backend)}

\subsubsection{Vercel (front)}
