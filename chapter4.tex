\chapter{\ChapterTitleWorkOrganization}
\label{sec:organizacja-pracy}

\section{Historia pomysłu projektu}

%%% jak było

\section{Charakterystyka i sposób realizacji projektu}

%%% bardzo ogólna wizja co chcemy mieć
%%% jak podzieliliśmy głowne częsci całej aplikacji
%%% core, backend, front, game-engine, thesis

\section{Weryfikacja postępów w trakcie realizacji}

\subsection{Milestone'y}
%%% najpierw base, pozniej lobby, pozniej gra a na koncu gra z ai
%%% scisle powiazane z epicami

\subsection{Spotkania z promotorem}
%%% to opisujmey w obu przypadkach \subsubsection{Microsoft Teams}

\subsection{Pracownia projektowa}
%%% to opisujmey w obu przypadkach \subsubsection{Microsoft Teams}


\section{Zastosowane praktyki i narzędzia organizacji pracy}

\subsection{Planowanie i projektowanie}

\subsubsection{Shortcut}

\subsubsection{Figma}


\section{Wykorzystywane technologie do realizacji projektu}

\subsection{Komunikacja wewnętrzna}

W trakcie tworzenia projektu potrzebna była szybka i efektywna
wymiana wiadomości oraz możliwość współpracy w czasie rzeczywistym.
Wykorzystana została w tym celu platforma Discord~\cite{Discord},
która oferuje rozmowy głosowe, kanały tekstowe i udostępnianie
obrazu na żywo.


\subsection{System kontroli wersji}

Aby synchronizować postępy pracy i ich historię zastosowano
system kontroli wersji Git \cite{Git}. Wraz z nim, do przechowywania
repozytoriów z kodem aplikacji, jak i dokumentacji wykorzystano
platformę GitHub \cite{Github}.

\subsection{Tworzenie dokumentacji}

Do napisania dokumentacji aplikacji użyto języka \LaTeX~\cite{Latex}.
Przesyłając tekst pracy na platformę GitHub, skorzystano z~technologii
GitHub Actions \cite{GithubActions}, którą wykorzystano w celu
generacji dokumentu od razu po opublikowaniu zmian w~tekście pracy.
Umożliwiło to na dostęp do najnowszej wersji dokumentu dla opiekuna
pracy i~prowadzącego pracownię projektową, dzięki czemu była możliwa
ciągła weryfikacja postępów nad pracą.


\subsection{Tworzenie oprogramowania}

\subsubsection{VSCode}

\subsubsection{Produkty Jetbrains}

\subsubsection{Postman}


\subsection{Wdrażanie i testowanie}

\subsubsection{Github Actions (deploy \& tests)}

%%%\subsubsubsection{Konteneryzacja (repozytorium)}

\subsubsection{Oracle Cloud (backend)}

\subsubsection{Vercel (front)}
