\chapter{\ChapterTitleWorkOrganization}
\label{sec:organizacja-pracy}

W~tym rozdziale omawiane są początkowe idee związane z~formą aplikacji,
wizją jej funkcji oraz zarysem harmonogramu działań. Ponadto, przedstawiane
są narzędzia, które zespół wykorzystał do skutecznego zarządzania
i~realizacji projektu. Opisana jest również struktura organizacyjna
zespołu, jego role i~zadania w kontekście tworzenia aplikacji.

Ponadto, zawiera informacje dotyczące przygotowania narzędzi do
weryfikacji postępu pracy, z~myślą o~osobach nadzorujących proces
realizacji projektu.

\section{Historia pomysłu projektu}

%%% jak było

\section{Charakterystyka i sposób realizacji projektu}

%%% bardzo ogólna wizja co chcemy mieć
%%% jak podzieliliśmy głowne częsci całej aplikacji
%%% core, backend, front, game-engine, thesis

\section{Weryfikacja postępów w trakcie realizacji}

\subsection{Etapy realizacji projektu}
%%% najpierw base, pozniej lobby, pozniej gra a na koncu gra z ai
%%% scisle powiazane z epicami

\subsection{Spotkania z promotorem}
%%% to opisujmey w obu przypadkach \subsubsection{Microsoft Teams}

\subsection{Pracownia projektowa}
%%% to opisujmey w obu przypadkach \subsubsection{Microsoft Teams}


\section{Wykorzystywane narzędzia do realizacji i organizacji pracy}

Do realizacji projektu zostało wykorzystane wiele narzędzi i~technologii
dostępnych za darmo lub jako produkty open source, czyli takie,
których właściciel praw autorskich przyznaje użytkownikom prawa
do swobodnego używania, modyfikacji i~udostępniania oprogramowania.

Skorzystano także z~produktów komercyjnych, które zostały udostępnione
bezpłatnie w~celach naukowych.

\subsection{Planowanie i projektowanie}

Przed rozpoczęciem tworzenia pierwszego kodu aplikacji niezbędne było
opracowanie ogólnego planu i~pierwotnej wizji projektu. Zdecydowano
się na staranne przygotowanie strategii działania oraz precyzyjnego
zarysu celów i~funkcji, które mają być zrealizowane w~ramach projektu.
To etap planowania stanowił fundament dla późniejszej implementacji
i~umożliwił efektywne kierowanie procesem tworzenia aplikacji.

Co więcej, w~ramach tego wstępnego etapu, zwrócono uwagę na
opracowanie wizualizacji aplikacji. Starano się osiągnąć jasny
zarys wyglądu aplikacji webowej poprzez stworzenie wstępnych projektów
interfejsu użytkownika. Ta wizualizacja odegrała kluczową rolę
w~procesie projektowania, umożliwiając zespołowi uzyskanie wyobrażenia
o~finalnym wyglądzie aplikacji i~jednocześnie dostosowywanie projektu
do oczekiwań i~potrzeb użytkowników, jak i~wzorowaniu się na utworzonej
wizualizacji.

\subsubsection{Shortcut}

Praktycznie zawsze w~trakcie tworzenia oprogramowania wymagane jest
narzędzie lub platforma udostępniająca możliwość tworzenia zadań,
zależności między nimi oraz dzielenia na większe zbiory zadań.

Zdecydowano się na narzędzie Shortcut \cite{Shortcut}, które oferuje
wszystkie potrzebne funkcjonalności. Pozwala na podział zadań na tzw.
epici -- elementy pracy zawierające wystarczająco dużo zadań do realizacji,
których nie da się ukończyć w~krótkim czasie. Dla każdego milestone'a
utworzono odpowiedni epic, dzięki czemu możliwe było prezentowanie
kolejnych części i~kluczowych etapów procesu tworzenia aplikacji.

\begin{figure}[h]
    \centering
    \includegraphics[width=\textwidth]{img/shortcut/shortcut_backlog.png}
    \caption{Backlog z zadaniami}
\end{figure}

\begin{figure}[h]
    \centering
    \includegraphics[width=\textwidth]{img/shortcut/shortcut_epic.png}
    \caption{Epic dotyczący systemu lobby wraz z~zadaniami i~ich statusami}
\end{figure}

\FloatBarrier


\subsubsection{Figma}

Do utworzenia wstępnej wizualizacji aplikacji użyto narzędzie Figma \cite{Figma}.
Służy ono do projektowania graficznego, takich prac jak szkielety stron
internetowych, prototypy interfejsów użytkownika czy też otwartych przestrzeni
do zarządzania i~planowania.

Warto wspomnieć, że tak jak w~\ref{fig:figma_login} lub
\ref{fig:figma_userflow} wykorzystano Figmę do stworzenia mocków i~schematów.

\begin{figure}[h]
    \centering
    \includegraphics[width=\textwidth]{img/schematy/milestones.png}
    \caption{Szczegółowy podział milestone'ów na zadania}
\end{figure}

\FloatBarrier


\subsection{Komunikacja wewnętrzna}

W~trakcie tworzenia projektu potrzebna była szybka i~efektywna
wymiana wiadomości oraz możliwość współpracy w czasie rzeczywistym.
Wykorzystana została w~tym celu platforma Discord~\cite{Discord},
która oferuje rozmowy głosowe, kanały tekstowe i~udostępnianie
obrazu na żywo.

\subsection{System kontroli wersji}

Aby synchronizować postępy pracy i~ich historię zastosowano
system kontroli wersji Git \cite{Git}. Wraz z~nim, do przechowywania
repozytoriów z kodem aplikacji, jak i~dokumentacji wykorzystano
platformę GitHub \cite{Github}.

\subsection{Tworzenie dokumentacji}

Do napisania dokumentacji aplikacji użyto języka \LaTeX~\cite{Latex}.
Przesyłając tekst pracy na platformę GitHub, skorzystano z~technologii
GitHub Actions \cite{GithubActions}, którą wykorzystano w celu
generacji dokumentu natychmiast po opublikowaniu zmian w~tekście pracy.
Umożliwiło to na dostęp do najnowszej wersji dokumentu dla opiekuna
pracy i~prowadzącego pracownię projektową, dzięki czemu była możliwa
ciągła weryfikacja postępów nad pracą.


\subsection{Tworzenie oprogramowania}

Rozwój oprogramowania stanowił kluczowy i~niezbędny element w~realizacji
założeń projektowych.
Naszym celem było stworzenie funkcjonalnej aplikacji, która integruje
odmienne komponenty, wykorzystujące różne technologie.
Aby sprostać złożonym wymaganiom projektu, skorzystaliśmy z~szerokiego
spektrum narzędzi.

\subsubsection{VSCode}

Visual Studio Code \cite{VSCode}, dalej opisywany jako VSCode,
to zaawansowany edytor kodu stworzony
przez Microsoft, który stał się jednym z~najpopularniejszych narzędzi
wśród programistów.
VSCode wspiera wiele języków programowania i~znaczników od JavaScript,
TypeScript, Python, PHP, C++ po HTML, CSS, JSON i~wiele innych.
Do tego oferuje zaawansowane funkcje auto uzupełniania kodu, informacje
o~typach, podpowiedzi dotyczące parametrów funkcji, info o~definicjach
i~dokumentację.
Dużą zaletą VSCode jest również możliwość personalizacji. Dostępne są
tysiące rozszerzeń, które pozwalają na dostosowanie edytora do własnych
potrzeb. Użycie tego edytora znacznie usprawniło rozwój wizualny aplikacji.

\subsubsection{Produkty Jetbrains}

Produkty JetBrains \cite{JetBrains}, w~tym dedykowane środowiska
programistyczne (IDE) dla poszczególnych języków i~ekosystemów,
takich jak Java, Python czy PHP.
Są one cenione za zaawansowane funkcje auto uzupełniania i~dogłębnej
analizy kodu. Dostarczają one również potężnych narzędzi do refaktoryzacji,
co znacząco ułatwia zarządzanie i~optymalizację projektu.

PyCharm \cite{PyCharm}, specjalizujący się w~Pythonie, był pomocny podczas tworzenia
modułu logiki gry w~brydża, a~także w~początkowej fazie konstruowania
serwera z~użyciem FastAPI.

Z~kolei IntelliJ IDEA \cite{Intellij}, ze wsparciem dla języka
Scala, znalazł zastosowanie w~procesie rozwijania ostatecznej wersji
architektury serwerowej. Wykorzystanie tych narzędzi znacząco przyczyniło
się do efektywności naszego procesu deweloperskiego.

\subsubsection{Postman}

Postman \cite{Postman} to narzędzie do testowania API, które umożliwia, głównie
programistom i testerom, skuteczne zarządzanie, tworzenie, udostępnianie
i~testowanie żądań HTTP. Jest to popularne środowisko do tworzenia
i~wykonywania zapytań HTTP, zwłaszcza w~kontekście testowania
endpointów RESTful API.

Postman oferuje intuicyjny interfejs graficzny, który ułatwia tworzenie,
wysyłanie i~analizowanie komunikatów sieciowych, a~także sprawdzanie
odpowiedzi serwera. Przydaje się w~wielu etapach procesu tworzenia
aplikacji, od projektowania, wczesnego testowania i~monitorowania API
w~środowisku produkcyjnym.

Przez cały proces tworzenia serwera dla aplikacji Postman był wykorzystywany
do wspomnianych wyżej celów. Było to narzędzie kluczowe w~zapewnieniu
sprawnego działania tej części systemu.


\subsection{Wdrażanie i testowanie}

\subsubsection{Github Actions (deploy \& tests)}

%%%\subsubsubsection{Konteneryzacja (repozytorium)}

\subsubsection{Oracle Cloud (backend)}

\subsubsection{Vercel (front)}
