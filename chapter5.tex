\chapter{\ChapterTitleResults}
\label{sec:wyniki-projektu}

\section{Weryfikacja realizacji wstępnych założeń}

W~pierwszym etapie planowania projektu utworzony został ogólny
plan, który opisywał oddzielne systemy i~elementy
zleconej aplikacji (Rys. \ref{fig:figma_strategicplan}).
Uwzględnia on nie tylko wymagania zdefiniowane w~rozdziale drugim
\nameref{sec:zakres-funkcjonalnosci}, ale także potrzebne do ich
realizacji wymagania techniczne.

\begin{figure}[h!]
    \centering
    \includegraphics[width=0.9\textwidth]{img/schematy/milestones.png}
    \caption{Szczegółowy podział milestone'ów na zadania}
    \label{fig:figma_strategicplan}
\end{figure}

Opierając się na powyższym planie, można stwierdzić, że udało się
zrealizować wszystkie wymagania sprecyzowane przez klienta.
Jedynym elementem, który został zrealizowany alternatywnie,
zgodnie z~założeniami zagrożeń implementacji
(\nameref{sec:analiza_zagrozen}), był wirtualny asystent. Szczegóły
dotyczące tej części projektu zostały opisane w~sekcji dotyczącej
asystenta \nameref{subsubsec:mocai}.

\section{Przegląd zrealizowanych funkcjonalności}

\subsection{Interfejs aplikacji}

Interfejs aplikacji był tworzony z~uwagą na postawione wymagania funkcjonalne. 
Zapewnia on użytkownikowi dostęp do wszystkich funkcji niezbędnych do korzystania 
z~aplikacji. Widoki aplikacji są w~znacznym stopniu podobne do szkiców wykonanych na etapie 
rozpoczęcia prac. Różnią się nieznacznie stylem ze względu na użycie innych narzędzi 
do ich utworzenia. Są też bardziej szczegółowe od szkiców, ponieważ nie planowano 
rozmieszczenia najdrobniejszych elementów interfejsu, takich jak przyciski akcji hosta 
w lobby czy oznaczenia pozycji graczy na ekranie gry. Przykładowe porównania końcowych 
rezultatów ze szkicami są przedstawione na rysunkach \ref{fig:compare_lobby} i 
\ref{fig:compare_game}.

\begin{figure}[h!]
    \centering
    \subfloat[Ekran lobby]{
      \includegraphics[width=0.45\textwidth]{img/widoki/lobby.png}
    }%
    \hspace*{0.5cm}
    \subfloat[Szkic lobby]{
      \includegraphics[width=0.45\textwidth]{img/figma-szkic/3.png}
    }%
    \caption{Porównanie ekranu lobby z początkowym szkicem}
    \label{fig:compare_lobby}
  \end{figure}
  
  \FloatBarrier

\begin{figure}[h!]
    \centering
    \subfloat[Ekran lobby]{
      \includegraphics[width=0.45\textwidth]{img/figma-szkic/1.png}
      \label{fig:figma_login2}
    }%
    \hspace*{0.5cm}
    \subfloat[Szkic lobby]{
      \includegraphics[width=0.45\textwidth]{img/figma-szkic/2.png}
    }%
    \caption{Porównanie ekranu gry z początkowym szkicem}
    \label{fig:compare_game}
  \end{figure}
  
\FloatBarrier
 
\subsubsection{Tworzenie i dołączenie do gry}

Na stronie głównej aplikacji użytkownik może utworzyć nową rozgrywkę 
poprzez naciśnięcie przycisku "Create game" lub dołączyć do istniejącej poprzez 
wpisanie kodu gry otrzymanego od innego użytkownika i~naciśnięcie przycisku "Join".

\subsubsection{Lobby}

Strona lobby służy do zebrania graczy przed rozpoczęciem gry i~ustawienia ich 
na opdowiednich pozycjach. Widok zawiera cztery pozycje z~pseudonimami graczy,
którzy się na nich znajdują. Host jest oznaczony ikoną korony, znajdującą się nad nazwą 
jego pozycji.

\subsubsection{Host}

Zgodnie z~założeniami, host ma możliwość wyrzucania graczy z~lobby, zmieniania ich pozycji 
oraz obsadzania pustych miejsc graczami-asystentami. Ponadto, host może oddać swoje 
uprawnienia innemu graczowi-użytkownikowi. Te akcje może wykonywać poprzez użycie odpowiednich 
przycisków znajdujących się nad pozycjami graczy.

\subsubsection{Użytkownik}

Użytkownik ma możliwość w~dowolnym momencie opuścić lobby, naciskając przycisk "Leave", 
oraz zgłosić gotowość do rozpoczęcia rozgrywki, poprzez naciśnięcie przycisku "Ready". 
Dodatkowo, może skopiować kod gry za pomocą przycisku "Copy ID" i~przesłać go innemu 
użytkownikowi. Warto zaznaczyć, że kod nie jest widoczny w~żadnym momencie, co może być 
przydatne dla graczy udostępniających ekran podczas rozgrywki, aby zabezpieczyć się przed 
nieproszonymi uczestnikami.

\subsection{Gra w brydża}

\subsubsection{Licytacja}
\subsubsection{Rozgrywka}


\subsection{Wirtualny asystent}
%%% przeprowadzone testy aplikacji na wybranej grupie osób (do wywalenia?)

\subsubsection{Moc AI}
\label{subsubsec:mocai}
%%% testy/badania czy AI wygrywa

\subsubsection{Możliwości rozwoju modelu}
%%% czy da sie rozszerzac o kolejne elementy gry brydza
%%% mozna sie odwolac do dalszych perspektyw (na ten moment w PR)

\subsection{Ułatwienia dostępności}

Oprócz omówionych kluczowych funkcjonalności, w~ramach rozwoju
aplikacji wprowadzono także dodatkowe elementy, których
potrzebę przedstawiono w~podrozdziale dotyczącym wymagań
niefunkcjonalnych. Głównych ich celem było zapewnienie
wygody i~prostego korzystania z~aplikacji, nawet podczas
pierwszego użytkowania. Poniżej przedstawione rozwiązania powstały
z~inicjatywy zespołu. Zostały one zaakceptowane przez klienta,
realizując wymagania dostępności i~użyteczności.

\subsubsection{Intuicyjność interfejsu}

Aby aplikacja była intuicyjna dla każdego użytkownika,
zastosowano ikony wprost kojarzące się z~wykonywaną
przez nie akcją. Zastosowano wyróżniające się kolory wśród palety
aplikacji w~celu podkreślenia danych czynności wykonywanych
przez użytkownika (Rys. \ref{fig:host_actions_ui}). Kontrastowe barwy mają zwrócić uwagę na
skutki danej akcji. Przykładowo jaskrawy czerwony kolor
kojarzący się przemocą lub impulsywnością wraz z~ikoną
przedstawiającą "$\times$"\xspace odpowiada za wyrzucenie gracza
z~sesji. Zastosowanie wspomnianych technik przestrzega
użytkownika przed przedwczesnym kliknięciem, ale także kojarzą
się one z~negatywną czynnością.

\begin{figure}[h!]
  \centering
  \includegraphics[width=\textwidth]{img/widoki/host_actions.png}
  \caption{Akcje hosta lobby}
  \label{fig:host_actions_ui}
\end{figure}

\FloatBarrier

\subsubsection{Responsywny układ aplikacji}

Zgodnie z~wymogiem dostępności interfejsy aplikacji powinny
być funkcjonalne również na urządzeniach o~niewielkich
rozmiarach ekranu. Wszystkie strony zostały zaprojektowane
tak, aby umożliwić wygodne korzystanie zarówno na urządzeniach
stacjonarnych, jak i~mobilnych (Rys. \ref{fig:responsive_ui}). Aplikacja dynamicznie
dostosowuje się w~zależności od aktualnego rozmiaru okna
przeglądarki.

Minimalna przewidziana
szerokość ekranu wynosi 280 pikseli, dzięki czemu wspierane
są także starsze urządzenia o~niewielkiej rozdzielczości
ekranu.

\begin{figure}[h!]
  \centering
  \includegraphics[width=\textwidth]{img/widoki/desktop_mobile.png}
  \caption{Tryb mobilny i desktopowy aplikacji}
  \label{fig:responsive_ui}
\end{figure}

\FloatBarrier

\subsubsection{Motywy jasny i ciemny}

Wygląd aplikacji został zrealizowany w~dwóch trybach --
jasnym i~ciemnym (Rys. \ref{fig:themes_ui}). Użyto w~tym celu dostępnych w~bibliotece
palet kolorów, ale także zdefiniowano własne, aby zachować
motyw kolorystyczny aplikacji. W~zależności od aktualnie
wybranego motywu kolory zmieniały swój odcień.

% \begin{figure}[h!]
%   \centering
%   \includegraphics[width=\textwidth, draft=true]{example-image}
%   \caption{Motywy jasny i ciemny aplikacji}
%   \label{fig:themes_ui}
% \end{figure}

% \FloatBarrier

\begin{figure}[h!]
  \centering
  \begin{minipage}[b]{0.45\textwidth}
    \centering
    \includegraphics[width=\textwidth]{img/widoki/light.png}
  \end{minipage}%
  \hspace*{0.5cm}
  \begin{minipage}[b]{0.45\textwidth}
    \centering
    \includegraphics[width=\textwidth]{img/widoki/dark.png}
  \end{minipage}
  \caption{Motywy jasny i ciemny aplikacji}
  \label{fig:themes_ui}
\end{figure}

\FloatBarrier

\section{Dalsze perspektywy rozwoju projektu}

\section{Podsumowanie}


