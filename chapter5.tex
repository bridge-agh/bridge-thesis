\chapter{\ChapterTitleResults}
\label{sec:wyniki-projektu}

\section{Weryfikacja realizacji wstępnych założeń}
%%% bardzo ogolne
%%% czy udało nam sie zrealizowac projekt i wszystkie zalozenia
%%% z wczesniejszych rozdzialow

%%% opisac funkcje produktu czy zostaly zrealizowane/w jaki sposob
\subsection{funkcja1}
\subsection{funkcja2}
\subsection{funkcja3}


\section{Interfejs użytkownika}

\subsection{Tryb desktop i mobilny}

\subsection{Strona główna -- tworzenie i dołączenie do gry}

\subsection{Lobby}
\subsubsection{Host}
\subsubsection{Użytkownik}

\subsection{Rozgrywka}

\subsection{Profil użytkownika}

\subsection{Porównanie pokrycia z planem interfejsu}
%%% ze udalo nam sie pokryc plany z figmy


\section{Testy użyteczności}
%%% przeprowadzone testy aplikacji na wybranej grupie osób
%%% testy czy AI wygrywa

\subsection{Wygoda i intuicja}
\subsection{Moc AI}


\section{Dalsze perspektywy rozwoju projektu}

Aplikacja do gry w~brydża została stworzona głównie
na potrzeby asystenta, aby można było go wykorzystać w~grze
poprzez sieć Internet. Tworzona była w~technologiach, które
w~naszej opinii są popularne, a~ich rozwój jest stale wspierany.
Dzięki temu możliwy jest przyszły rozwój lub przebudowa aktualnej
wersji aplikacji.
W~związku z~tym zaproponowano potencjalne funkcjonalności rozszerzające
możliwości asystenta i~aplikacji, które
mogłyby być rozwinięte w~przyszłości. \\

Jedną z~ważniejszych funkcji jest wprowadzenie wsparcia do
przeprowadzenia pełnej rozgrywki gry. Na ten moment możliwe jest
rozegranie jednego rozdania. Przy wprowadzeniu pełnej gry należałoby
utworzyć zapisy wyników z~każdej partii, w~których skład wchodzą rozdania,
a~także dostosować asystenta
do uwzględniania wyników gry\footnote{Uwzględnienie wyników gry jest
    rozumiane jako dodanie wyników gry do obserwacji modelu asystenta.}
podczas podejmowania decyzji. \\

Również istotną propozycją, która z~pewnością podniesie poziom
współpracy asystenta z~graczem, jest wsparcie języków licytacji\footnote{
    język licytacji -- odzywki licytacyjne mające przekazać informacje
    partnerowi, którym przypisane
    są ustalone informacje o~posiadanych kartach.
} (systemów licytacji).
Aktualnie asystent podejmuje decyzje podczas licytacji bez znajomości
języków. Oznacza to, że nie rozumie odzywek gracza w~stosowanym przez
niego systemie licytacyjnym. \\

Dodatkową funkcjonalnością, która rozszerza zastosowanie asystenta, jest
wykorzystanie go do analizy rozgrywek w~czasie rzeczywistym lub
ich historycznych zapisów. Dzięki temu gracz mógłby uzyskać podpowiedzi
o~możliwie najlepszych ruchach na podstawie aktualnego stanu gry.
W~przypadku gier już odbytych mógłby także sprawdzić, w~jakich etapach
zostały popełnione błędy lub wykonano najlepsze z~możliwych ruchów.

Również można rozwinąć analizę przeprowadzoną przez asystenta, aby
podpowiedzi do gry były zintegrowane z~modelem językowym. Dzięki
takiemu rozwiązaniu gracz mógłby korzystać z~chatu dostępnego
z~poziomu aplikacji, aby zadać pytanie o~aktualnie przeprowadzanej
rozgrywce lub na temat zasad gry w~brydża. Model językowy zapewniłby
zrozumiałe dla gracza odpowiedzi i~w~razie potrzeby mógłby bardziej
szczegółowo sprezycować odpowiedź na prośbę gracza.

\section{Podsumowanie}


