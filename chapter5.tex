\chapter{\ChapterTitleResults}
\label{sec:wyniki-projektu}

\section{Weryfikacja realizacji wstępnych założeń}

W~pierwszym etapie planowania projektu utworzony został ogólny
plan, który opisywał oddzielne systemy i~elementy
zleconej aplikacji (Rys. \ref{fig:figma_strategicplan}).
Uwzględnia on nie tylko wymagania zdefiniowane w~rozdziale drugim
\nameref{sec:zakres-funkcjonalnosci}, ale także potrzebne do ich
realizacji wymagania techniczne.

\begin{figure}[h!]
  \centering
  \includegraphics[width=0.9\textwidth]{img/schematy/milestones.png}
  \caption{Szczegółowy podział milestone'ów na zadania}
  \label{fig:figma_strategicplan}
\end{figure}

Opierając się na powyższym planie, można stwierdzić, że udało się
zrealizować wszystkie wymagania sprecyzowane przez klienta.
Jedynym elementem, który został zrealizowany alternatywnie,
zgodnie z~założeniami zagrożeń implementacji
(\nameref{sec:analiza_zagrozen}), był wirtualny asystent. Szczegóły
dotyczące tej części projektu zostały opisane w~sekcji dotyczącej
asystenta \nameref{subsubsec:mocai}.

\section{Przegląd zrealizowanych funkcjonalności}

\subsection{Interfejs aplikacji}

\subsubsection{Tworzenie i dołączenie do gry}
\subsubsection{Lobby}
\subsubsection{Host}
\subsubsection{Użytkownik}


\subsection{Gra w brydża}

Utworzony został prosty interfejs gry, który zapewnia
wymagane elementy gry jak system licytacji i~widoczne
karty odpowiednich graczy. Ograniczono się głównie do
zaimplementowania tych akcji, jakie mógłby wykonać także
wirtualny asystent.

\subsubsection{Licytacja}

Podczas licytacji użytkownik może na bieżąco obserwować
podejmowane kontrakty przez innych graczy. Niedostępne akcje
są blokowane, co ułatwia wybór odpowiedniego kontraktu.
Wskazywany jest również gracz, który w~danym momencie
mógłby wygrać licytację, gdy trzech kolejnych graczy
pasuje.

\begin{figure}[h!]
  \centering
  \includegraphics[width=0.9\textwidth]{img/widoki/bidding.png}
  \caption{Interfejs gry podczas licytacji}
  \label{fig:bidding}
\end{figure}

\FloatBarrier

\subsubsection{Rozgrywka}

W~ekranie rozgrywki znalazły się takie elementy jak widok
kart każdego z~graczy, w~tym widoczne są awersy kart
użytkownika i~dziadka (lub partnera, gdy użytkownik jest
dziadkiem). Litery znajdujące się przy kartach określają
kierunek, na którym się znajdują. Podświetlany jest
kierunek tego gracza, który aktualnie wykonuje ruch.
Na środku ekranu znajdują się zagrane karty tworzące lewę.
Wszystkie karty i~ich zagrania są płynnie animowane,
dzięki czemu użytkownicy mogą doświadczyć gry
w~podobny sposób jak przy prawdziwym stole.

\begin{figure}[h!]
  \centering
  \includegraphics[width=0.9\textwidth]{img/widoki/game.png}
  \caption{Interfejs gry podczas rozgrywki}
  \label{fig:bidding}
\end{figure}

\FloatBarrier

\subsection{Wirtualny asystent}
%%% przeprowadzone testy aplikacji na wybranej grupie osób (do wywalenia?)

\subsubsection{Moc AI}
\label{subsubsec:mocai}
%%% testy/badania czy AI wygrywa

\subsubsection{Możliwości rozwoju modelu}
%%% czy da sie rozszerzac o kolejne elementy gry brydza
%%% mozna sie odwolac do dalszych perspektyw (na ten moment w PR)

\subsection{Ułatwienia dostępności}

Oprócz omówionych kluczowych funkcjonalności, w~ramach rozwoju
aplikacji wprowadzono także dodatkowe elementy, których
potrzebę przedstawiono w~podrozdziale dotyczącym wymagań
niefunkcjonalnych. Głównych ich celem było zapewnienie
wygody i~prostego korzystania z~aplikacji, nawet podczas
pierwszego użytkowania. Poniżej przedstawione rozwiązania powstały
z~inicjatywy zespołu. Zostały one zaakceptowane przez klienta,
realizując wymagania dostępności i~użyteczności.

\subsubsection{Intuicyjność interfejsu}

Aby aplikacja była intuicyjna dla każdego użytkownika,
zastosowano ikony wprost kojarzące się z~wykonywaną
przez nie akcją. Zastosowano wyróżniające się kolory wśród palety
aplikacji w~celu podkreślenia danych czynności wykonywanych
przez użytkownika (Rys. \ref{fig:host_actions_ui}). Kontrastowe barwy mają zwrócić uwagę na
skutki danej akcji. Przykładowo jaskrawy czerwony kolor
kojarzący się przemocą lub impulsywnością wraz z~ikoną
przedstawiającą "$\times$"\xspace odpowiada za wyrzucenie gracza
z~sesji. Zastosowanie wspomnianych technik przestrzega
użytkownika przed przedwczesnym kliknięciem, ale także kojarzą
się one z~negatywną czynnością.

\begin{figure}[h!]
  \centering
  \includegraphics[width=0.9\textwidth]{img/widoki/host_actions.png}
  \caption{Akcje hosta lobby}
  \label{fig:host_actions_ui}
\end{figure}

\FloatBarrier

\subsubsection{Responsywny układ aplikacji}

Zgodnie z~wymogiem dostępności interfejsy aplikacji powinny
być funkcjonalne również na urządzeniach o~niewielkich
rozmiarach ekranu. Wszystkie strony zostały zaprojektowane
tak, aby umożliwić wygodne korzystanie zarówno na urządzeniach
stacjonarnych, jak i~mobilnych (Rys. \ref{fig:responsive_ui}). Aplikacja dynamicznie
dostosowuje się w~zależności od aktualnego rozmiaru okna
przeglądarki.

Minimalna przewidziana
szerokość ekranu wynosi 280 pikseli, dzięki czemu wspierane
są także starsze urządzenia o~niewielkiej rozdzielczości
ekranu.

\begin{figure}[h!]
  \centering
  \includegraphics[width=0.9\textwidth]{img/widoki/desktop_mobile.png}
  \caption{Tryb mobilny i desktopowy aplikacji}
  \label{fig:responsive_ui}
\end{figure}

\FloatBarrier

\subsubsection{Motywy jasny i ciemny}

Wygląd aplikacji został zrealizowany w~dwóch trybach --
jasnym i~ciemnym (Rys. \ref{fig:themes_ui}). Użyto w~tym celu dostępnych w~bibliotece
palet kolorów, ale także zdefiniowano własne, aby zachować
motyw kolorystyczny aplikacji. W~zależności od aktualnie
wybranego motywu kolory zmieniały swój odcień.

% \begin{figure}[h!]
%   \centering
%   \includegraphics[width=\textwidth, draft=true]{example-image}
%   \caption{Motywy jasny i ciemny aplikacji}
%   \label{fig:themes_ui}
% \end{figure}

% \FloatBarrier

\begin{figure}[h!]
  \centering
  \begin{minipage}[b]{0.45\textwidth}
    \centering
    \includegraphics[width=\textwidth]{img/widoki/light.png}
  \end{minipage}%
  \hspace*{0.5cm}
  \begin{minipage}[b]{0.45\textwidth}
    \centering
    \includegraphics[width=\textwidth]{img/widoki/dark.png}
  \end{minipage}
  \caption{Motywy jasny i ciemny aplikacji}
  \label{fig:themes_ui}
\end{figure}

\FloatBarrier

\section{Dalsze perspektywy rozwoju projektu}

Aplikacja do gry w~brydża została stworzona głównie
na potrzeby asystenta, aby można było go wykorzystać w~grze
poprzez sieć Internet. Tworzona była w~technologiach, które
w~naszej opinii są popularne, a~ich rozwój jest stale wspierany.
Dzięki temu możliwy jest przyszły rozwój lub przebudowa aktualnej
wersji aplikacji.
W~związku z~tym zaproponowano potencjalne funkcjonalności rozszerzające
możliwości asystenta i~aplikacji, które
mogłyby być rozwinięte w~przyszłości. \\

Jedną z~ważniejszych funkcji jest wprowadzenie wsparcia do
przeprowadzenia pełnej rozgrywki gry. Na ten moment możliwe jest
rozegranie jednego rozdania. Przy wprowadzeniu pełnej gry należałoby
utworzyć zapisy wyników z~każdej partii, w~których skład wchodzą rozdania,
a~także dostosować asystenta
do uwzględniania wyników gry\footnote{Uwzględnienie wyników gry jest
  rozumiane jako dodanie wyników gry do obserwacji modelu asystenta.}
podczas podejmowania decyzji. Istotne w~tym przypadku
byłoby również rozszerzenie interfejsu gry, aby mógł zawierać
dodatkowe informacje, jak na przykład aktualny wynik gry lub
historię licytacji. \\

Również istotną propozycją, która z~pewnością podniesie poziom
współpracy asystenta z~graczem, jest wsparcie języków licytacji\footnote{
  język licytacji -- odzywki licytacyjne mające przekazać informacje
  partnerowi, którym przypisane
  są ustalone informacje o~posiadanych kartach.
} (systemów licytacji).
Aktualnie asystent podejmuje decyzje podczas licytacji bez znajomości
języków. Oznacza to, że nie rozumie odzywek gracza w~stosowanym przez
niego systemie licytacyjnym. \\

Dodatkową funkcjonalnością, która rozszerza zastosowanie asystenta, jest
wykorzystanie go do analizy rozgrywek w~czasie rzeczywistym lub
ich historycznych zapisów. Dzięki temu gracz mógłby uzyskać podpowiedzi
o~możliwie najlepszych ruchach na podstawie aktualnego stanu gry.
W~przypadku gier już odbytych mógłby także sprawdzić, w~jakich etapach
zostały popełnione błędy lub wykonano najlepsze z~możliwych ruchów.

Również można rozwinąć analizę przeprowadzoną przez asystenta, aby
podpowiedzi do gry były zintegrowane z~modelem językowym. Dzięki
takiemu rozwiązaniu gracz mógłby korzystać z~chatu dostępnego
z~poziomu aplikacji, aby zadać pytanie o~aktualnie przeprowadzanej
rozgrywce lub na temat zasad gry w~brydża. Model językowy zapewniłby
zrozumiałe dla gracza odpowiedzi i~w~razie potrzeby mógłby bardziej
szczegółowo sprezycować odpowiedź na prośbę gracza.

\section{Podsumowanie}


