\chapter{\ChapterTitleResults}
\label{sec:wyniki-projektu}

\section{Weryfikacja realizacji wstępnych założeń}
%%% bardzo ogolne
%%% czy udało nam sie zrealizowac projekt i wszystkie zalozenia
%%% z wczesniejszych rozdzialow

%%% opisac funkcje produktu czy zostaly zrealizowane/w jaki sposob
\subsection{funkcja1}
\subsection{funkcja2}
\subsection{funkcja3}

\subsection{Porównanie pokrycia z planem interfejsu}
% porownanie z figmą jaki powstal system (u nas udal sie caly)


\section{Przegląd zrealizowanych funkcjonalności}

\subsection{Interfejs aplikacji}

\subsubsection{Tworzenie i dołączenie do gry}
\subsubsection{Lobby}
\subsubsection{Host}
\subsubsection{Użytkownik}


\subsection{Gra w brydża}

\subsubsection{Licytacja}
\subsubsection{Rozgrywka}


\subsection{Wirtualny asystent}
%%% przeprowadzone testy aplikacji na wybranej grupie osób

\subsubsection{Moc AI}
%%% testy/badania czy AI wygrywa


\subsection{Ułatwienia dostępności}

\subsubsection{Intuicyjność interfejsu}
% intuicyjne ikony stosowane obszernie w sieci

\subsubsection{Responsywny układ aplikacji}
\subsubsection{Motywy jasny i ciemny}


\section{Dalsze perspektywy rozwoju projektu}

\section{Podsumowanie}


