\chapter{\ChapterTitleResults}
\label{sec:wyniki-projektu}

\section{Weryfikacja realizacji wstępnych założeń}
%%% bardzo ogolne
%%% czy udało nam sie zrealizowac projekt i wszystkie zalozenia
%%% z wczesniejszych rozdzialow

%%% opisac funkcje produktu czy zostaly zrealizowane/w jaki sposob
\subsection{funkcja1}
\subsection{funkcja2}
\subsection{funkcja3}

\subsection{Porównanie pokrycia z planem interfejsu}
% porownanie z figmą jaki powstal system (u nas udal sie caly)


\section{Przegląd zrealizowanych funkcjonalności}

\subsection{Interfejs aplikacji}

\subsubsection{Tworzenie i dołączenie do gry}
\subsubsection{Lobby}
\subsubsection{Host}
\subsubsection{Użytkownik}


\subsection{Gra w brydża}

\subsubsection{Licytacja}
\subsubsection{Rozgrywka}


\subsection{Wirtualny asystent}
%%% przeprowadzone testy aplikacji na wybranej grupie osób

\subsubsection{Moc AI}
%%% testy/badania czy AI wygrywa


\subsection{Ułatwienia dostępności}

Oprócz omówionych kluczowych funkcjonalności, w~ramach rozwoju
aplikacji wprowadzono także dodatkowe elementy, których
potrzebę przedstawiono w~podrozdziale dotyczącym wymagań
niefunkcjonalnych. Głównych ich celem było zapewnienie
wygody i~prostego korzystania z~aplikacji, nawet podczas
pierwszego użytkowania. Poniżej przedstawione rozwiązania powstały
z~inicjatywy zespołu. Zostały one zaakceptowane przez klienta,
realizując wymagania dostępności i~użyteczności.

\subsubsection{Intuicyjność interfejsu}

Aby aplikacja była intuicyjna dla każdego użytkownika,
zastosowano ikony wprost kojarzące się z~wykonywaną
przez nią akcją. Zastosowano wyróżniające się kolory wśród palety
aplikacji w~celu podkreślenia danych czynności wykonywanych
przez użytkownika (Rys. \ref{fig:host_actions_ui}). Kontrastowe barwy mają zwrócić uwagę na
skutki danej akcji. Przykładowo jaskrawy czerwony kolor
kojarzący się przemocą lub impulsywnością wraz z~ikoną
przedstawiającą "$\times$"\xspace odpowiada za wyrzucenie gracza
z~sesji. Zastosowanie wspomnianych technik przestrzega
użytkownika przed przedwczesnym kliknięciem, ale także kojarzą
się one z~negatywną czynnością.

\begin{figure}[h!]
  \centering
  \includegraphics[width=\textwidth, draft=true]{example-image}
  \caption{Akcje hosta lobby}
  \label{fig:host_actions_ui}
\end{figure}

\FloatBarrier

\subsubsection{Responsywny układ aplikacji}

Zgodnie z~wymogiem dostępności interfejsy aplikacji powinny
być funkcjonalne również na urządzeniach o~niewielkich
rozmiarach ekranu. Wszystkie strony zostały zaprojektowane
tak, aby umożliwić wygodne korzystanie zarówno na urządzeniach
stacjonarnych, jak i~mobilnych (Rys. \ref{fig:responsive_ui}). Aplikacja dynamicznie
dostosowuje się w~zależności od aktualnego rozmiaru okna
przeglądarki.

Minimalna przewidziana
szerokość ekranu wynosi 280 pikseli, dzięki czemu wspierane
są także starsze urządzenia o~nie wielkiej rozdzielczości
ekranu.

\begin{figure}[h!]
  \centering
  \includegraphics[width=\textwidth, draft=true]{example-image}
  \caption{Tryb mobilny i desktopowy aplikacji}
  \label{fig:responsive_ui}
\end{figure}

\FloatBarrier

\subsubsection{Motywy jasny i ciemny}

Wygląd aplikacji został zrealizowany w~dwóch trybach --
jasnym i~ciemnym (Rys. \ref{fig:themes_ui}). Użyto w~tym celu dostępnych w~bibliotece
palet kolorów, ale także zdefiniowano własne, aby zachować
motyw kolorystyczny aplikacji. W~zależności od aktualnie
wybranego motywu kolory zmieniały swój odcień.

\begin{figure}[h!]
  \centering
  \includegraphics[width=\textwidth, draft=true]{example-image}
  \caption{Motywy jasny i ciemny aplikacji}
  \label{fig:themes_ui}
\end{figure}

\FloatBarrier

\section{Dalsze perspektywy rozwoju projektu}

\section{Podsumowanie}


