\documentclass[polish]{aghengthesis}

\usepackage[utf8]{inputenc}
\usepackage{url}
\usepackage{subfig}
\usepackage{float}
\usepackage{tabularx}
\usepackage{ragged2e}
\usepackage{booktabs}
\usepackage{multirow}
\usepackage{grffile}
\usepackage{indentfirst}
\usepackage{caption}
\usepackage{listings}
\usepackage[ruled,linesnumbered,lined]{algorithm2e}
\usepackage[bookmarks=false]{hyperref}
\usepackage{placeins}

\hypersetup{colorlinks,
  linkcolor=blue,
  citecolor=blue,
  urlcolor=blue}

\usepackage[svgnames]{xcolor}
\usepackage{inconsolata}

\usepackage{csquotes}
\DeclareQuoteStyle[quotes]{polish}
  {\quotedblbase}
  {\textquotedblright}
  [0.05em]
  {\quotesinglbase}
  {\fixligatures\textquoteright}
\DeclareQuoteAlias[quotes]{polish}{polish}

\usepackage[nottoc]{tocbibind}

\usepackage[
style=numeric,
sorting=nyt,
isbn=false,
doi=true,
url=true,
backref=false,
backrefstyle=none,
maxnames=10,
giveninits=true,
abbreviate=true,
defernumbers=false,
backend=biber]{biblatex}
\addbibresource{main.bib}

\lstset{
  %language=Python, %% PHP, C, Java, etc.
  basicstyle=\ttfamily\footnotesize,
  backgroundcolor=\color{gray!5},
  commentstyle=\it\color{Green},
  keywordstyle=\color{Red},
  stringstyle=\color{Blue},
  numberstyle=\tiny\color{Black},    
  % morekeywords={TestKeyword},
  % mathescape=true,
  escapeinside=`',
  frame=single, %shadowbox, 
  tabsize=2,
  rulecolor=\color{black!30},
  title=\lstname,
  breaklines=true,
  breakatwhitespace=true,
  framextopmargin=2pt,
  framexbottommargin=2pt,
  extendedchars=false,
  captionpos=b,
  abovecaptionskip=5pt,
  keepspaces=true,            
  numbers=left,                    
  numbersep=5pt,                  
  showspaces=false,                
  showstringspaces=false,
  showtabs=false,
  tabsize=2
}

\SetAlgorithmName{\LangAlgorithm}{\LangAlgorithmRef}{\LangListOfAlgorithms}
\newcommand{\listofalgorithmes}{\tocfile{\listalgorithmcfname}{loa}}

\renewcommand{\lstlistingname}{\LangListing}
\renewcommand\lstlistlistingname{\LangListOfListings}

\renewcommand{\lstlistoflistings}{\begingroup
\tocfile{\lstlistlistingname}{lol}
\endgroup}

% Definicje nowych rodzajów kolumn w tabeli
\newcolumntype{Y}{>{\small\centering\arraybackslash}X}
%\newcolumntype{b}{>{\hsize=1.6\hsize}Y}
%\newcolumntype{m}{>{\hsize=.6\hsize}Y}
%\newcolumntype{s}{>{\hsize=.4\hsize}Y}

\captionsetup[figure]{skip=5pt,position=bottom}
\captionsetup[table]{skip=5pt,position=top}

%%%%%%%%%%%%%%%%%%%%%%%%%%%%%%%%%%%%%%%%%%%%%%%%%%%%%%%%%%%%%%%%%%%%%%%%%%%%%%%

\author{Szymon Idec, Jakub Karbowski, Karol Śliwa}

\titlePL{Wirtualny asystent gry w brydża}
\titleEN{A virtual bridge game assistant}

\fieldofstudy{Informatyka}

%\typeofstudies{Stacjonarne}

\supervisor{dr inż.\ Maciej Woźniak}

\date{\the\year}

%%%%%%%%%%%%%%%%%%%%%%%%%%%%%%%%%%%%%%%%%%%%%%%%%%%%%%%%%%%%%%%%%%%%%%%%%%%%%%%

\begin{document}

\maketitle

\tableofcontents

%%%%%%%%%%%%%%%%%%%%%%%%%%%%%%%%%%%%%%%%%%%%%%%%%%%%%%%%%%%%%%%%%%%%%%%%%%%%%%%

\chapter{\ChapterTitleProjectVision}
\label{sec:cel-wizja}

%%%

\section{Charakterystyka problemu}

Gry karciane, takie jak brydż, cieszą się dużą popularnością na całym świecie.
Jednakże wiele osób ma trudności w~znalezieniu odpowiednich partnerów do gry,
szczególnie w przypadku osób początkujących, które nie posiadają jeszcze
szerokiego kręgu znajomych zainteresowanych tą grą. W~takiej sytuacji, osoby
chcące nauczyć się gry w~brydża lub doskonalić swoje umiejętności, często
zniechęcają się do dalszego uczestnictwa w~rozgrywkach.

Ponadto, brydż jest grą wymagającą nie tylko umiejętności logicznego myślenia
i~strategii, ale także zdolności do skutecznej komunikacji i~współpracy
z~partnerem. Wielu początkujących graczy nie jest w~stanie zagrać pełnej
i~satysfakcjonującej gry, gdyż nie posiadają odpowiedniego doświadczenia
i~umiejętności. Dodatkowo niektórzy mogą nie być w stanie znaleźć
odpowiedniego partnera posiadającego podobny styl gry lub
rozumiejącego stosowaną strategię.

W~związku z~powyższymi problemami zidentyfikowano potrzebę opracowania
aplikacji do gry w brydża, która umożliwiłaby grę z~wirtualnym asystentem
na poziomie umiejętności dostosowanym do użytkownika, zapewniającym satysfakcjonującą
rozgrywkę oraz możliwość nauki i~doskonalenia umiejętności gry w~brydża.

%%%

\section{Motywacja projektu}

\begin{figure}
    \centering
    \includegraphics[width=0.9\textwidth]{img/brydz-platformy/bridgebase.png}
    \caption{Rozgrywka brydża na platformie Bridge Base}
    \label{fig:bridge-base}
\end{figure}

\begin{figure}
    \centering
    \subfloat[Główny ekran rozgrywki]{
        \includegraphics[width=.4\textwidth]{img/brydz-platformy/neuralplay1.jpg}
        \label{fig:neural-play-1}
    }%
    \hspace*{0.5cm}
    \subfloat[Historia bieżącej rozgrywki]{
        \includegraphics[width=.4\textwidth]{img/brydz-platformy/neuralplay2.jpg}
        \label{fig:neural-play-2}
    }%
    \caption{Rozgrywka brydża w aplikacji Bridge by NeuralPlay}
    \label{fig:neural-play}
\end{figure}

Wirtualny asystent może stanowić rozwiązanie dla tych problemów. Poprzez
zintegrowanie sztucznej inteligencji z~brydżem, aplikacja będzie w~stanie
zapewnić użytkownikom możliwość
rozgrywania partii z~wirtualnym partnerem, który będzie reagował na decyzje
gracza i~pomagał mu w~rozwijaniu jego umiejętności.

Dzięki temu projektowi, użytkownicy będą mieli możliwość prostszego i~bardziej
efektywnego nauczania się gry w~brydża. Lepszy dostęp odpowiednich partnerów
do gry pozwoli na intensywny i~satysfakcjonujący rozwój ich
umiejętności oraz~pasji. Asystent umożliwi naukę
i~doskonalenie umiejętności w~grze, a~także będzie w~stanie dostosować
się do odpowiedniego poziomu wybranego przez użytkownika.
\newline

Obecnie na rynku istnieje kilka analogicznych rozwiązań, jak na przykład
aplikacje internetowe Funbridge \cite{funbridge}, Bridge Base \cite{bridgebase},
BridgeAce+ \cite{bridgeace} oraz mobilna wersja Bridge by NeuralPlay
\cite{neuralplay}. W~przypadku BridgeAce+, użytkownik może korzystać wyłącznie
z~opcji nauki brydża w pojedynkę, gdzie partnerem i~przeciwnikami są boty
wykorzystujące sztuczną inteligencję. Natomiast Funbridge pozwala graczom
rywalizować ze sobą przez Internet oraz uczyć się grając z~botem poprzez
przechodzenie kursu. Po zakończeniu rozgrywki, użytkownik może przeanalizować
swoje błędy, jednak dostęp do tej opcji wymaga wykupienia konta premium.
Bridge Base oferuje wiele trybów gry, zarówno przeciwko botom jak i~innym
użytkownikom, jednak jego minus stanowi mało intuicyjny interfejs graficzny
(Rys.~\ref{fig:bridge-base}). Aplikacja NeuralPlay oferuje pełną rozgrywkę w~brydża
z~wykorzystaniem zaprojektowanego AI, które zna wybrane metody licytowania.
Niestety, nawet rozgrywka przeciwko najtrudniejszemu poziomowi sztucznej
inteligencji, może być wygrana przez osobę mało doświadczoną.
\newline

Jak można zauważyć, istnieją już dostępne aplikacje umożliwiające naukę
i~grę w brydża, jednak mają one istotne wady. Brakuje rozwiązania, które
byłoby dostępne bez konieczności ponoszenia kosztów, umożliwiało naukę
i~rywalizację z~innymi użytkownikami oraz posiadało intuicyjny i~przyjazny
interfejs graficzny. Celem projektu jest połączenie większości tych funkcji
w~dostępnej za darmo aplikacji.



\FloatBarrier

\section{Wizja produktu}

Produkt będzie aplikacją webową umożliwiająca grę w~brydża,
bez konieczności posiadania znajomych do gry.
Partnera lub przeciwnika będzie mógł zastąpić wirtualny asystent.
Jego zadaniem jest zapewnienie odpowiedniego
dla użytkownika towarzysza, którego poziom jest zależny od wybranej
preferencji. Użytkownicy będą mogli dostosować trudność, aby asystent osiągnął
planowane przez nich cele, takie jak nauka, rozgrywka na podobnym poziomie
lub zostać wyzwaniem dla doświadczonych zawodników.

Aplikacja będzie również umożliwiała grę w~brydża z~udziałem jednego lub więcej użytkowników,
którzy będą mogli połączyć się przez Internet, dzięki centralnemu serwerowi.
Zamierzane jest zaprojektowanie jej w~taki sposób, aby była intuicyjna i~łatwa
w~obsłudze dla wszystkich użytkowników, zarówno początkujących, jak
i~doświadczonych.

%%%

\section{Stos technologiczny}

Aplikacja internetowa zostanie napisana w~frameworku React \cite{React}.
Zdecydowaliśmy się na ten wybór, ze względu na dużą popularność tej
technologii. W~2022 roku serwis Stack Overflow \cite{StackOverflow} przeprowadził ankietę dotyczącą między innymi wykorzystywanych technologii
webowych \cite{React-stack}. Aż 42.62\% respondentów wybrało React,
zajmując 2 miejsce w~rankingu. Jako że wykorzystuje on język Javascript,
który według wspomnianej ankiety jest najpopularniejszym językiem programowania,
dostępna jest olbrzymia baza bibliotek i~sprawdzonych rozwiązań. Za wizualną
część projektu będzie odpowiedzialny framework Tailwind CSS \cite{Tailwind}.
Oferuje on gotowe klasy CSS, które pozwalają na szybkie tworzenie responsywnych
i~estetycznych interfejsów. Część serwerowa obsługująca sesje gier zostanie
napisana w~języku Python \cite{Python} za pomocą biblioteki FastAPI
\cite{FastAPI}. Funkcjonalność związana z~obsługą użytkowników i~gromadzenia
danych będzie wykorzystywała usługę Google Firebase \cite{Firebase}.

Wirtualny asystent zostanie zrealizowany jako osobna biblioteka udostępniająca
własne API. Backend obsługujący sesje gier, po dołączeniu biblioteki asystenta,
będzie mógł wysłać do API asystenta aktualny stan gry, aby otrzymać
odpowiedź zawierającą analizę tego stanu, między innymi
prawdopodobieństwo wygrania każdej z~par graczy oraz listę
najlepszych ruchów, jakie mogą być wykonane przez następnego gracza.

Na podstawie wstępnych badań, wybrane zostało 5~algorytmów AI,
które będą rozważane do zastosowania w~projekcie asystenta.
W~kolejności od najbardziej złożonego do
najprostszego w implementacji są to:
\begin{enumerate}
    \item \textbf{Algorytm Regularized Nash Dynamics \cite{doi:10.1126/science.add4679}}\\
          Algorytm oparty na teorii gier, który wykorzystuje sieć neuronową do
          predykcji najlepszego ruchu. Istnieje dowód formalny, że algorytm
          zbiega do równowagi Nasha, czyli optymalnej strategii.

    \item \textbf{Algorytm AlphaZero \cite{AlphaZeroPaper} połączony z algorytmem IS-MCTS \cite{6203567}}\\
          Algorytm AlphaZero jest wariantem algorytmu Monte Carlo Tree Search,
          który wykorzystuje sieć neuronową do oceny stanu gry.
          Algorytm Information Set MCTS (IS-MCTS) jest modyfikacją algorytmu MCTS, która pozwala na
          przeszukiwanie drzewa stanów gry o~niepełnej informacji.
          Możliwe jest połączenie tych dwóch algorytmów, aby uzyskać
          Information Set AlphaZero.

    \item \textbf{Implementacja algorytmu IS-MCTS z biblioteki OpenSpiel \cite{LanctotEtAl2019OpenSpiel}}\\
          Biblioteka OpenSpiel zawiera implementację algorytmu IS-MCTS,
          który może być wykorzystany do gry w~brydża.

    \item \textbf{Program AI do brydża GIB \cite{Ginsberg1999GIBST}}\\
          GIB jest jednym ze starszych programów AI do brydża. Możliwe jest
          jego bezpośrednie wykorzystanie w~projekcie.

    \item \textbf{Algorytm Pure Monte Carlo Tree Search \cite{pmcts}}\\
          Algorytm równoważny MCTS dla maksymalnej głębokości drzewa równej 1.
\end{enumerate}


%%%

\section{Analiza zagrożeń}
\label{sec:analiza_zagrozen}

Gra w~brydża jest jedną z~najtrudniejszych gier strategicznych na świecie.
Do tej pory nie powstał żaden system AI grający na mistrzowskim poziomie
uwzględniający pełną wersję gry z~licytacją \cite{Bethe2021AdvancesIC}.
Implementacja gracza AI o~wystarczająco wysokim poziomie umiejętności może
być znacznym wyzwaniem. W~literaturze zostało opisane wiele metod AI do brydża
\cite{Zhang2019DesignAD,Zhang2022TheSO,Zhang2022AIEB,Ginsberg1999GIBST}
co sugeruje, że jest to temat dalej otwarty i~ciągle rozwijany.
Przedstawiliśmy 5~różnych możliwości implementacji AI.
W~razie problemów z~implementacją jednej z~nich, pozostałe
zostaną wykorzystane jako plany awaryjne.

Ważnym elementem każdej gry online jest niezawodność systemu
backend oraz jego odporność na awarie.
Aplikacja musi być również odporna na problemy sieciowe.
Platforma Firebase zapewnia nam mechanizmy zabezpieczające
przed problemami po stronie klienta, połączenia internetowego
oraz samego backendu Firebase, który jest redundantnie
rozproszony po całym świecie.
Samo hostowanie strony internetowej zostanie zrealizowane
usługą Firebase Hosting, która również zapewnia wysoką
niezawodność.
Bezpieczeństwo aplikacji będzie zarządzane za pomocą
usługi Firebase Authentication.

Powyżej wymienione usługi są dostępne w darmowym planie.
Podczas pracy nad projektem może dojść do sytuacji, że
zostaną one przekształcone w~plan płatny.
Jeśli dostępne platformy chmurowe staną się niedostępne,
zastosowany zostanie własny serwer działający
w~oparciu o~komputer osobisty lub mikrokomputer, np. Raspberry Pi \cite{RPi}.

% needs latex magic
\begin{table}[h]
    \centering
    \begin{tabularx}{\textwidth}{|p{5.5cm}|Y|Y|}
        \hline
        \textbf{Zagrożenie}                                    & \textbf{Prawdopodobieństwo wystąpienia} & \textbf{Zagrożenie dla projektu} \tabularnewline[0.2cm]
        \hline
        Problemy w implementacji AI                            & Wysokie                                 & Średnie (podano plany awaryjne)  \tabularnewline[0.2cm]
        Nie osiągniecie poziomu mistrzowskiego przez asystenta & Wysokie                                 & Niskie                           \tabularnewline[0.3cm]
        Niska niezawodność produktu                            & Niskie                                  & Średnie                          \tabularnewline[0.1cm]
        Niska niezawodność produktu                            & Niskie                                  & Średnie                          \tabularnewline[0.1cm]
        \hline
    \end{tabularx}
    \caption{Analiza zagrożeń}
    \label{tab:zagrozenia}
\end{table}

%%%

\newpage

\section{Szkic aplikacji}

Zostały przygotowane szkice interfejsu aplikacji
(Rys.~\ref{fig:figma_login}--\ref{fig:figma_game}).

\begin{figure}[h!]
    \centering
    \subfloat[Ekran logowania]{
        \includegraphics[width=0.45\textwidth]{img/figma-szkic/1.png}
        \label{fig:figma_login}
    }%
    \hspace*{0.5cm}
    \subfloat[Ekran główny]{
        \includegraphics[width=0.45\textwidth]{img/figma-szkic/2.png}
    }%
    \\
    \subfloat[Ekran lobby]{
        \includegraphics[width=0.45\textwidth]{img/figma-szkic/3.png}
    }%
    \hspace*{0.5cm}
    \subfloat[Ekran gry]{
        \includegraphics[width=0.45\textwidth]{img/figma-szkic/4.png}
        \label{fig:figma_game}
    }%
    \caption{Szkice interfejsu aplikacji}
\end{figure}

\FloatBarrier

\section{Słownik pojęć}

\begin{enumerate}
    \item Brydż -- karciana gra strategiczna rozgrywana przez dwie pary graczy,
    \item Użytkownik -- człowiek korzystający z aplikacji,
    \item Wirtualny asystent -- algorytm posiadający możliwość analizy rozgrywki w celu nauki, współpracy lub konkurencji z użytkownikiem,
    \item Gracz -- użytkownik lub wirtualny asystent biorący udział w rozgrywce,
    \item Partner -- drugi gracz grający w parze danego gracza,
    \item Przeciwnik -- jeden z graczy z pary grającej przeciwko danemu graczowi,
    \item Wirtualny partner -- partner będący wirtualnym asystentem,
    \item Wirtualny przeciwnik -- przeciwnik będący wirtualnym asystentem,
\end{enumerate}


\chapter{\ChapterTitleScope}
\label{sec:zakres-funkcjonalnosci}


\section{Kontekst użytkowania aplikacji}

\begin{figure}[b!]
  \centering
  \includegraphics[width=0.9\textwidth]{img/flow-aplikacji/user_flow.png}
  \caption{Schemat interakcji użytkownika z aplikacją}
  \label{fig:figma_userflow}
\end{figure}

\begin{figure}[p!]
  \centering
  \includegraphics[width=\textwidth]{img/flow-aplikacji/game_flow.png}
  \caption{Schemat interakcji użytkownika z aplikacją podczas rozgrywki brydża}
\end{figure}

Głównym zadaniem aplikacji jest oferowanie możliwości
rozgrywania gry w~brydża. Strona internetowa aplikacji ma być
przystosowana do dowolnego rozmiaru ekranu urządzenia, z~którego mógłby
korzystać użytkownik. Dzięki temu można skorzystać z~aplikacji, wykorzystując
telefon, komputer stacjonarny, laptop lub nawet telewizor. Wymagany
jest tylko dostęp do Internetu. \\


W~systemie aplikacji zdefiniowane są dwa typy użytkowników:
\begin{itemize}
  \item \textbf{anonimowy użytkownik} -- osoba niezalogowana, która nie
        posiada dostępu do funkcjonalności aplikacji związanych
        z~rozgrywką.

  \item \textbf{zalogowany użytkownik} -- zalogowana osoba posiadająca konto
        w~aplikacji, mająca pełny dostęp do jej funkcjonalności.
\end{itemize}

Funkcjonalności dostępne dla poszczególnych użytkowników:
\begin{itemize}
  \item \textbf{anonimowy użytkownik}:
        \begin{itemize}
          \item rejestracja nowego konta w~systemie aplikacji.
          \item logowanie do konta utworzonego w~systemie aplikacji.
        \end{itemize}

  \item \textbf{zalogowany użytkownik}:
        \begin{itemize}
          \item funkcjonalności użytkownika anonimowego.
          \item tworzenie i~dołączanie do rozgrywek w brydża --
                gdy użytkownik założy lub zostanie jedynym
                użytkownikiem lobby, to staje się jego
                hostem.
          \item zarządzanie lobby -- może decydować o~graczach znajdujących się wraz
                z~nim w~lobby. Może decydować o~ich pozycji lub ich wyrzucić
                ze wspólnej sesji\footnote{Sesja oznacza to samo co lobby, czyli
                  abstrakcję łączącą wielu graczy i~asystentów w~obrębie jednej
                  rozgrywki}.
                Może także uzupełnić wolne miejsca wirtualnym asystentem
                z~określeniem jego poziomu trudności.
        \end{itemize}
\end{itemize}

\FloatBarrier

\section{Przypadki użycia}

\subsection{Rejestracja i logowanie do aplikacji}

Aby uzyskać dostęp do większości funkcjonalności aplikacji, wymagane
jest posiadanie konta. Anonimowy użytkownik może je utworzyć, klikając
opcję \textbf{"Register"} w~nagłówku strony. Po kliknięciu użytkownik
zostanie przekierowany do formularza rejestracyjnego.
Do utworzenia konta wymagane jest podanie własnego
pseudonimu, adresu e-mail oraz hasła. Aby uzyskać dostęp do utworzonego
konta, należy kliknąć \textbf{"Log in"} w~nagłówku strony, po czym
w~formularzu podać dane wykorzystanie podczas rejestracji.

\begin{figure}[hbt!]
  \centering
  \includegraphics[width=0.9\textwidth]{img/schematy/login.png}
  \caption{Schemat logowania i~rejestracji użytkownika}
\end{figure}

W~przypadku nieprawidłowo podanych danych podczas rejestracji
lub logowania, wystąpienia błędów wynikających z~połączenia internetowego lub
niedostępnego serwerem autentykacji, użytkownik otrzyma odpowiednią
informację na ekranie.

\FloatBarrier


\subsection{Lobby}

\subsubsection{Tworzenie lobby}

Rozpoczęcie gry w~brydża jest dostępne z~poziomu lobby. Aby utworzyć
lobby, należy kliknąć \textbf{"Create lobby"} na głównym panelu
aplikacji. Przekieruje ono użytkownika do nowego lobby, którego staje
się administratorem. Do utworzonego lobby zostanie wygenerowany
unikalny identyfikator, zwany dalej kodem. Użytkownik będzie mógł go skopiować
i~udostępnić, np. znajomym, aby mogli oni do niego dołączyć.


\subsubsection{Dołączanie do lobby}

Gdy użytkownik otrzyma kod do lobby od innego użytkownika, może
do niego dołączyć, poprzez główny panel aplikacji. Należy wkleić
kod w~odpowiednie pole i~klikając opcję \textbf{"Join"}
dołączyć do przypisanego do kodu lobby.

\begin{figure}[hbt!]
  \centering
  \includegraphics[width=0.9\textwidth]{img/schematy/create_join_lobby.png}
  \caption{Schemat tworzenia i dołączania do lobby}
\end{figure}

\FloatBarrier


\subsubsection{Zarządzanie lobby}

Jeżeli użytkownik jest hostem lub został jedynym
użytkownikiem w~lobby, może on zarządzać innymi graczami.
Posiada on następujące możliwości:
\begin{itemize}
  \item zamiana pozycji w lobby -- dwie wybrane pozycje są zamieniane miejscami.
        Może być to gracz lub puste miejsce.
  \item usunięcie gracza z lobby -- gracz opuszcza lobby i~nie będzie
        mógł już ponownie do niego dołączyć.
  \item zajęcie wolnej pozycji jako wirtualny asystent -- wybrana pozycja
        w~grze będzie kontrolowana przez asystenta
        i~uczestniczyć w~rozgrywce jako partner lub przeciwnik.
  \item zamknięcie lobby -- jeżeli administrator jest ostatnim
        użytkownikiem w~lobby, opuszczenie go spowoduje jego automatyczne
        zamknięcie.
        %  \item zmiana statusu lobby na publiczne/prywatne -- powoduje
        %         dostępność rozgrywki rozpoczętej przez lobby w~panelu

\end{itemize}

\begin{figure}[hbt!]
  \centering
  \includegraphics[width=0.9\textwidth]{img/schematy/manage_lobby.png}
  \caption{Schemat zarządzania lobby}
\end{figure}

\FloatBarrier


\subsubsection{Zgłoszenie gotowości}
Użytkownik po dołączeniu do lobby może zgłosić gotowość do rozgrywki poprzez
naciśnięcie przycisku "\textbf{Ready}". Kiedy wszyscy użytkownicy w lobby zgłoszą
gotowość, zaczyna się gra.


\subsection{Gra w brydża}

\subsubsection{Licytacja}

\begin{figure}[hbt!]
  \centering
  \includegraphics[width=0.9\textwidth]{img/schematy/bid.png}
  \caption{Schemat zgłoszenie odzywki}
\end{figure}

Na etapie początkowym gry, uczestnicy mają możliwość zgłaszania odzywek poprzez
aktywację przycisków odpowiadających ich wyborom. Mogą licytować kontrakty, pasować lub
deklarować kontrę i~rekontrę. Pierwszy licytujący jest losowo
wybranym graczem. Faza licytacji dobiega końca, gdy odzywka "pas" zostanie zagrana
trzy razy z rzędu.


\FloatBarrier


\subsubsection{Zagrywanie kart}

Po zakończeniu fazy licytacji, gracz podczas swojej kolejki może zagrać kartę,
przesuwając kursor na nią i~klikając myszką. Dodatkowo, gracz będący partnerem dziadka
ma możliwość zagrywania kart w~jego imieniu podczas jego tury.

\begin{figure}[hbt!]
  \centering
  \includegraphics[width=0.9\textwidth]{img/schematy/play_card.png}
  \caption{Schemat zagrania karty przez użytkownika}
\end{figure}

\FloatBarrier


\subsubsection{Zastąpienie gracza}
W przypadku, gdy jeden z użytkowników opuści grę, jego miejsce zostanie automatycznie
zajęte przez wirtualnego asystenta. Jednakże, jeśli wszyscy użytkownicy
zdecydują się opuścić rozgrywkę, to zostanie ona zakończona.

% \subsection{Obserwowanie rozgrywek}

% Gdy jakaś rozgrywka w~brydża została rozpoczęta i~ma ona status
% publiczny, jest możliwe obejrzenie jej z~panelu \textbf{Watch}.
% Z~listy publicznych rozgrywek należy wybrać interesującą i~kliknąć
% ikonę oka. 
% \end{itemize}

% \begin{figure}[h]
%   \centering
%   \includegraphics[width=\textwidth]{example-image-a}
%   \caption{}
% \end{figure}

% \FloatBarrier


\section{Specyfikacja wymagań funkcjonalnych}

Aplikacja nie tylko powinna być dobrze zaprojektowana i~wygodna
w~użyciu, ale także funkcjonalna. Musi spełniać założone wymagania,
udostępniając odpowiednie funkcjonalności, żeby była przydatna dla
jej użytkowników. Nie da się używać wirtualnego asystenta do gry
w~brydża, jeśli brakuje w nim algorytmu AI, czy w~ogóle elementu
rozgrywki. Bez spełnienia kluczowych wymagań funkcjonalnych,
aplikacja praktycznie nie istnieje. W~tym podrozdziale przedstawiamy
te wymagania.


\subsection{Logowanie i rejestracja nowych użytkowników}
Żeby uzyskać dostęp do funkcjonalności naszej aplikacji, potrzebne
jest posiadanie konta i bycie zalogowanym. Z~tego względu koniecznością
jest umożliwienie użytkownikom utworzenia konta, logowania się na
nie i~możliwość późniejszego wylogowania. Formularz rejestracji wymaga
podania adresu e-mail, nazwy użytkownika i~hasła. Logowanie odbywa się
poprzez wprowadzenie swojego adresu e-mail i~hasła.


\subsection{Lobby}
W celu rozegrania partii brydża przeciwko innym graczom,
użytkownik musi mieć możliwość utworzenia własnej
rozgrywki oraz dołączenia do istniejącej.

\subsubsection{Utworzenie/Dołączenie do lobby}
Utworzenie, jak i~dołączenie do lobby jest dostępne z~widoku głównego aplikacji. Dołączenie do lobby
odbywa się poprzez wprowadzenie identyfikatora gry. Po utworzeniu lobby użytkownik może
wysłać jego identyfikator innym użytkownikom, zapraszając ich do dołączenia.

\subsubsection{Zarządzanie lobby}
Host powinien mieć możliwość dostosowania lobby według swoich preferencji przed
rozpoczęciem gry. Host ma prawo wyrzucić gracza z~lobby, zmienić jego pozycję lub przekazać
mu swoje uprawnienia. Dodatkowo, host powinien móc przydzielić wybrane pozycje
wirtualnym asystentom.


\subsection{Rozpoczęcie rozgrywki}
Kiedy wszystkie miejsca w lobby zostaną zajęte przez graczy, którzy zgłoszą gotowość do gry
lub są wirtualnymi asystentami, rozgrywka zostaje rozpoczęta.
Wszyscy użytkownicy zostają przeniesieni do ekranu gry.


\subsection{Rozgrywka}
Głównym celem i~podstawową funkcjonalnością naszej aplikacji jest
rozgrywka w brydża. Zaczyna się od fazy licytacji, po której
gracze wykładają karty zgodnie z~zasadami gry. Gra kończy się po rozegraniu 
trzynastu lew. Po skończonej rozgrywce użytkownikowi wyświetlany jest ekran z~wynikami.

\subsection{Wyświetlenie wyników}
Ważne jest, żeby użytkownicy po skończeniu partii mogli zobaczyć
podsumowanie całej rozgrywki. Użytkownikowi wyświetlana jest liczba 
zdobytych lew. Dzięki temu użytkownicy mogą na koniec porównać rezultat rozgrywki 
ze swoimi założeniami z fazy licytacji i lepiej oceniać swoje możliwości w~przyszłych rozgrywkach.


\subsection{Opuszczenie rozgrywki}
Użytkownik może nie tylko wyjść z~lobby, ale także w~dowolnym momencie
opuścić rozgrywkę. W~takim przypadku miejsce użytkownika zastępuje
wirtualny asystent.



% TODO

\section{Specyfikacja wymagań niefunkcjonalnych}
W naszym projekcie aplikacji do gry w~brydża
istotą jest zapewnienie nie tylko wszystkich potrzebnych wymagań
funkcjonalnych, ale także wygodnej i~intuicyjnej. Niewiele osób
będzie chętnie korzystać z~aplikacji, choćby miała ona rozbudowane
spektrum funkcji, jeśli jej działanie będzie niestabilne i~pozbawione
intuicyjności. W~tym podrozdziale przedstawiamy wymagania
niefunkcjonalne aplikacji, które są równie ważne, jak zdefiniowane
wcześniej wymagania funkcjonalne.


\subsection{Dostępność}
Zdecydowano, aby tworzona aplikacja była dostępna w~formie webowej.
Dzięki temu będzie on dostępny zarówno dla użytkowników
Windows, MacOS, Linux, jak i~innych systemów operacyjnych posiadających
przeglądarkę wspierającą najnowsze standardy.
Kluczowe jest dostosowanie interfejsów tak,
aby były wygodne i~funkcjonalne nawet na urządzeniach o~niewielkich
rozmiarach ekranu, umożliwiając płynne korzystanie z~aplikacji również
na urządzeniach mobilnych.


\subsection{Użyteczność}
Priorytetem aplikacji jest zapewnienie łatwości obsługi
i~zrozumiałości działania elementów interfejsu.
Istotne jest, aby interfejs użytkownika
był intuicyjny, estetyczny i~nie zniechęcał potencjalnych użytkowników
oraz nie utrudniał korzystania z~aplikacji. Ustalono, aby podczas
projektowania interfejsu użytkownika kierować się zasadami
dobrego UI/UX.
Za cel postanowiono stworzenie minimalistycznego i~uporządkowanego
interfejsu, który zapewniłby spójność na wszystkich podstronach aplikacji.
Wykorzystamy ikony o~jednakowym znaczeniu na wszystkich stronach,
co ułatwi nawigację. Dodatkowo zastosujemy stałą gamę starannie
dobranych kolorów dla elementów interfejsu, co pozwoli nam utrzymać
spójny styl podczas projektowania kolejnych widoków. Aplikacja
będzie oferować tryb jasny i~ciemny, aby użytkownicy mogli wygodnie
korzystać z~niej w~zależności od preferencji lub pory dnia.
Ponadto, wszystkie podstrony będą responsywne, co będzie oznaczać,
że elementy interfejsu będą zachowywać pełną funkcjonalność,
niezależnie od rozmiaru okna przeglądarki, na którym są wyświetlane.

\subsection{Niezawodność}
Wirtualny asystent do gry w~brydża ma być tworzony z~myślą o~nauce
i~doskonaleniu umiejętności, ale także konkurencji i~wzajemnym
rozgrywkom pomiędzy użytkownikami. Konieczne jest więc wprowadzenie logiki
gry z~największą dokładnością. Błędy w działaniu aplikacji podczas
rozgrywki mogłyby wpłynąć na wynik gry oraz wprowadzić nowych użytkowników
w~zakłopotanie, a~bardziej doświadczonych w~stan irytacji.
Niezwykle istotne jest również uniknięcie sytuacji, w~których
dochodziłoby do utraty połączenia lub braku synchronizacji pomiędzy
użytkownikami. Takie incydenty mogą zrujnować całą rozgrywkę i~zdecydowanie
obniżyć wartość jednej z~kluczowych funkcjonalności aplikacji, jaką
jest możliwość rozgrywania partii między użytkownikami.
Również błędy implementacyjne logiki gry sprawiłyby problemy
z~działaniem asystenta. Nie jest to dopuszczalne, zważając na to,
że asystent ma być konkurencyjnym i~sprawnie działającym rozwiązaniem.

Pierwszym krokiem w~stronę niezawodności aplikacji było wybranie
odpowiednich technologii do jej realizacji. Narzędzia Next.js
oraz FastAPI służące do budowy aplikacji webowych zostały
przez nas uznane, jako godne zaufania, a~zarazem są
popularne na całym świecie. Usługa Google Firebase odciąża nas z~implementacji
funkcjonalności gromadzenia danych i~obsługi użytkowników od podstaw.
Mając odpowiedni stos technologiczny, przystąpiliśmy do implementacji
kolejnych funkcjonalności.


\chapter{\ChapterTitleRealizationAspects}
\label{sec:wybrane-aspekty-realizacji}

Ta sekcja pracy skupia się na opisaniu najważniejszych aspektów
opracowanej aplikacji.
Jej celem jest szczegółowe opisanie wybranych
komponentów oraz przedstawienie czynników, które są najistotniejsze do
funkcjonowania całości systemu. Dodatkowo skoncentruje
się na prezentacji skutecznych rozwiązań dla wykrytych problemów.

\section{Architektura serwera}
%%% TODO expand basic info
Serwer napisany w języku Scala.


%%% jak działa lobby bo ciekawe

%%% utrata dostepow do endpointow w przypadku zmiany stanow lobby/gry



\subsection{Komunikacja z serwerem}
Wstępne założenia dotyczące infrastruktury aplikacji
serwerowej opierały się na wykorzystaniu języka Python
wraz z~frameworkiem FastAPI.

% TODO od tad
Z~czasem jednak, w~świetle
narastających wymagań i~doświadczeń zgromadzonych
podczas etapu tworzenia, nasz zespół podjął
strategiczną decyzję o~ewolucji stosu technologicznego.
% do tad do zmiany - "w~świetle
% narastających wymagań i~doświadczeń zgromadzonych
% podczas etapu tworzenia" totalnie wypada
Ostatecznie, struktura serwera została skonstruowana
w~dynamicznym języku Scala, korzystając z~potencjału
frameworka Akka, specjalizującego się w~programowaniu
współbieżnym oraz rozproszonym, bazującym na
zaawansowanym modelu aktorów.

Pierwotny model komunikacji między klientem a~serwerem
był zbudowany w~oparciu o~metodę \textbf{short polling},
%%% TODO jakiś słownik co znaczy short polling???
zaimplementowaną przy pomocy FastAPI. Ten sposób
w~praktycznym zastosowaniu ujawnił pewne ograniczenia.
Regularne zapytania wysyłane przez klienta w~celu
sprawdzenia stanu danych na serwerze prowadziły do
znacznych opóźnień w~aktualizacji interfejsu
użytkownika. Taka sytuacja stwarzała trudności nawet
przy realizacji bardzo prostych funkcji, takich jak
zmiana pozycji użytkowników w~przestrzeni lobby.

Wobec wyzwań, narzuconych przez ograniczenia wyżej wspomnianej metody, nasz
zespół rozważał migrację ku strategii \textbf{long polling}. Metoda ta,
w~odróżnieniu od konwencjonalnych technik polegających na ciągłym
wysyłaniu zapytań przez klienta w~celu odświeżenia danych, proponuje
utrzymanie otwartego kanału HTTP do czasu, aż serwer będzie miał
najnowsze informacje do przekazania. Mimo iż koncepcja ta wydawała się
obiecująca, powstały obawy związane z~koniecznością ciągłego monitorowania
potencjalnej utraty połączenia. Skłoniło nas do ostatecznej rezygnacji
z~wdrożenia tego pomysłu.

Implementacja frameworka Akka \cite{Akka} pozwoliła na ustanowienie stabilnego,
\textbf{reaktywnego} połączenia między klientem a~serwerem. Programowanie
reaktywne, koncentrujące się na płynnym przepływie danych i~ich propagacji,
umożliwiło aplikacji klienta natychmiastowo reagować na wszelkie zmiany. Jest
to szczególnie istotne w~kontekście naszej aplikacji, która charakteryzuje
się intensywną interakcją użytkownika, zwłaszcza podczas naszej kluczowej
funkcjonalności -- rozgrywki w~brydża. Zastosowanie tego podejścia
znacząco usprawniło koordynację sekwencji animacji ruchów poszczególnych
graczy.

%%% mozna opisac to pozniej
% \subsection{Testy}
% Framework Akka dostarcza także kompleksowe narzędzia do efektywnego
% testowania aktorów, w~tym symulowane środowisko czasu wykonania, które
% pozwala na dogłębne sprawdzenie zachowań asynchronicznych w~warunkach
% synchronicznych. Dzięki tym możliwościom możemy rozwijać architekturę
% serwera, mając pełne przekonanie co do jej stabilności i~niezawodności
% działania.

\section{Aplikacja webowa}
Aplikacja webowa została stworzona przy użyciu platformy Next.js
\cite{NextJS} oraz języka programowania Typescript \cite{Typescript},
będącego rozszerzeniem języka JavaScript o~statyczne typowanie.
Jednakże ze względu na problemy wydajnościowe rozgrywki
i~słabą responsywność interfejsu związaną z~połączeniem serwerowym
zdecydowano się na zmianę technologii odpowiedzialnych za te zadania.

Implementacja rozgrywki w~brydża została logicznie
odseparowana od reszty aplikacji, choć w~pełni integruje się z~samą
platformą i~korzysta z~wcześniej wspomnianych języków programowania.
Szczegółowy opis części znajduje się w~podrozdziale \nameref{subsec:silnik_gry}.

\subsection{Mechanizmy stanu i cyklu życia komponentów}
Biblioteka React oferuje narzędzie w~postaci \textbf{hooków},
które umożliwiają efektywne zarządzanie stanem i~cyklem życia komponentów.
Są to funkcje, które udostępniają stany obiektów, oferując do nich wgląd, automatyczną aktualizację
lub zmianę wartości stanów.

W~naszym projekcie, do obsługi zapytań klienta do serwera, zdecydowaliśmy się
skorzystać z~biblioteki SWR \cite{SWR}.
Jest to rozwiązanie, korzystające z~hooków, do
pobierania danych rozwiązujące wiele problemów implementacynych takich jak
cache'owanie i~deduplikacje (usuwanie redundantnych kopii) danych, ponawianie żądań
czy weryfikacja połączenia po odzyskaniu dostępu do sieci.

Aby jeszcze bardziej zoptymalizować proces, stworzyliśmy
również zestaw własnych, dedykowanych hooków. Takie podejście pozwoliło na
znaczną standaryzację logiki odpowiedzialnej za aktualizacje stanów przy pomocy
zapytań HTTP z~użyciem SWR. Przyczyniło się do zmniejszenia ilości
globalnych zmian kodu, gdyż zmiana implementacji nie zmieniła samego interfejsu
hooka. Funkcjonalność pozostała niezmienna i~oferowane stany miały taką samą strukturę.

%%% TODO - kod/zdjecia wybranych hook'ów (sam kod może miec dodatkowe komentarze
%%% opisujące jakies ciekawe cechy)

Jak zostało wspomniane we wcześniejszym podrozdziale, dotyczącym serwera aplikacji,
zrezygnowano ze wspomnianego rozwiązania biblioteki SWR na rzecz reaktywnego
połączenia z~serwerem. Wymusiło to zmianę implementacji wysyłania i~odbierania
zapytań do serwera. Logika
wykorzystania hooków pozostała taka sama, przez co komponenty interfejsu
zawierające wspomniane hooki nie musiały być aktualizowane po tej zmianie. \\

Poniżej przedstawione zostały hooki odpowiedzialne za aktualizacje
danych z~serwera, wykorzystując \textbf{Fetch API} do wysyłania pojedynczych
zapytań i~\textbf{WebSockety} do ustanawiania połączenia z~serwerem
i~ciągłą aktualizację oferowanego stanu przez serwer.

\begin{lstlisting}[language=JavaScript, caption=Hooki z Fetch API, label={lst:fetch-hooks}, captionpos=b]
export function useFetch<T, U>(fetcher: (request: T) => Promise<U>): FetchState<T, U> {
  const [data, setData] = useState<U | undefined>(undefined);
  const [loading, setLoading] = useState<boolean>(false);

  const trigger = useCallback((request: T) => {
    if (loading) return Promise.reject();
    setData(undefined);
    setLoading(true);
    return fetcher(request)
      .then(data => {
        setData(data);
        return data;
      })
      .finally(() => {
        setLoading(false);
      });
  }, [fetcher, loading]);

  return { trigger, data, loading };
}

async function createLobbyFetcher(unused: void): Promise<CreateLobbyResponse> {
  const token = await getIdToken();
  const res = await fetch($`$$\dollar${API_URL_SESSION_LOBBY}/create$`$, {
    method: "POST",
    headers: {
      "Content-Type": "application/json",
      "Authorization": $`$Bearer $\dollar${token}$`$
    },
  });
  if (!res.ok) return Promise.reject(res.statusText);
  return res.json();
}

export function useCreateLobby() {
  return useFetch(createLobbyFetcher);
}
\end{lstlisting}

\vspace*{0.5cm}

\begin{lstlisting}[language=JavaScript, caption=Hooki z WebSocketem, label={lst:websocket-hooks}, captionpos=b]
export function useSocket<T>(url: string | undefined): SocketState<T> {
  url = url?.replace(/^http/, "ws");

  const [data, setData] = useState<T | undefined>(undefined);
  const loading = data === undefined;

  const getAuthUrl = useCallback(() => {
    return getIdToken().then(token => {
      return $`$$\dollar${url}?access_token=$\dollar${token}$`$;
    });
  }, [url]);

  useWebSocket(url ? getAuthUrl : null, {
    shouldReconnect: () => true,
    onMessage: (event) => {
      setData(JSON.parse(event.data));
    },
    reconnectInterval: 500,
  });

  return { data, loading };
}  

export interface GetInfoResponse {
  sessionId: string
  hostId: string
  users: Player[]
  started: boolean
  gameState: GameState | null
}

export function useSessionInfo(): SocketState<GetInfoResponse> {
  return useSocket<GetInfoResponse>($`$$\dollar${API_URL_SESSION}/info$`$);
}

\end{lstlisting}

\subsection{Stylizacja interfejsu}
W~celu nadania stylów komponentom naszej aplikacji internetowej posłużono się
frameworkia Tailwind CSS. W~połączeniu z~biblioteką Daisy UI
\cite{DaisyUI}, która
w~pełni wspiera ten framework, możliwe było zastosowanie gotowych klas
stylistycznych. Oferują one nie tylko samą edycję pojedynczych elementów, ale też
utworzenie zaprojektowanych, w~pełni animowanych obiektów.

Ze względu na intuicyjne nazewnictwo klas stylizujących możliwe było szybkie
projektowanie interfejsu użytkownika i~wyeliminowało
potrzebę bezpośredniego zarządzania plikami CSS. Jest to szczególnie
przydatne, jeżeli wykorzystywany jest React, ze względu na jego naturalny podział
elementów na osobne komponenty. Wykorzystanie tego podejścia znacznie przyspieszyło
proces rozwoju części wizualnej aplikacji.


\section{Silnik gry}
\label{subsec:silnik_gry}

Silnik wizualny rozgrywki do gry w~brydża jest jednym z~kluczowych
elementów aplikacji. Podczas tworzenia go skupiono się, aby zaoferować
wydajność i~pełną integrację z~aplikacją webową.


\subsection{Framework Three.js}
Podczas próby implementacji środowiska do gry, powstał problem z~zarządzaniem
elementami kart i~ich przemieszczaniem po ekranie użytkownika. Animacje,
takie jak ruch kart, z~wykorzystaniem wyłącznie komponentów React
okazały się nadmiernie pracochłonne i~nieefektywne. Użytkownik
mobilny miał problemy z~ładowaniem kart
i~ich płynnym animowaniem. Jest to spowodowane inną architekturą procesorów
mobilnych w~porównaniu do rozwiązań desktopowych. Nie są one przystosowane
do wykonania najmocniejszych obliczeń, ale oferują niskie użycie energii. \\

Wymusiło to rezygnację z~rozwiązania wykorzystującego jedynie elementy HTML.
Dokonano zmiany na bardzo efektywny framework umożliwiający renderowanie
środowiska i~elementów 3D - Three.js
\cite{ThreeJS}.

Posiada on wszystkie potrzebne funkcje, których brakło w~naszym początkowym
podejściu. Udostępnia on:
\textit{canvas} -- którego zadaniem jest zdefiniowanie
obszaru, na którym znajdują się wyświetlane elementy;
\textit{przestrzeń 3D} - w której możemy umieszczać
karty i~przemieszczać niezależnie od struktury komponentów;
\textit{kamerę} -- pozwalającą na wykorzystanie perspektywy
w~środowisku 3D, przez co gracze mogą mieć wizualne
odczucie realnej rozgrywki gry w~brydża.

%%% TODO zdjęcie planszy gry 3D (orbithelper w gamescene)

Dodatkowo framework ten wybrany został ze względu na swoją niezwykłą wydajność.
Do renderowania elementów wykorzystuję on potencjał WebGL \cite{WebGL}.
Jest to API
w~języku JavaScript, które jest standardem webowym wykorzystywanym
w~przeglądarkach. Umożliwia on wykorzystanie akceleracji sprzętowej kart
graficznych, które są przystosowane do przetwarzania fizyki, zdjęć
i~efektów wizualnych. Kluczowe jest, że stosując wspomnianą akcelerację,
wykorzystujemy większe możliwości sprzętowe urządzeń mobilnych, jak
i~desktopowych, które nie były one w~pełni wykorzystywane przy zastosowaniu
jedynie elementów HTML.
Zastosowanie tego rozwiązania
drastycznie zwiększyło efektywność i~płynność silnika gry na tych urządzeniach.
Przykładowo badając skok wydajności, porównując pierwotny silnik z~obecnym,
na urządzeniu mobilnym, różnica jest na poziomie od około 15 wyświetlanych
klatek na sekundę aż do
ponad 140\footnote{
  Porównanie zostało wykonane na urządzeniu
  Samsung Galaxy S10+ z procesorem Samsung Exynos 9820.
}.

\subsection{Integracja silnika z Reactem}
Three.js oprócz wspomnianych zalet jest też wspierany w~postaci komponentów React.
Rozwiązania, które są wykorzystane w~aplikacji to React Three Fiber
\cite{ReactThreeFiber} i~React drei \cite{ReactDrei}. Oba oferują
wspomniane komponenty, jak i~dodatkowe moduły
wspierające prace Three.js w~środowisku React. Uniknięto w~ten sposób korzystania
z~JavaScript'a, a~wykorzystano oferowane hooki i~komponenty



\section{AI}

\section{System wdrażania do środowiska deweloperskiego}

%%% vercel jak to dziala z githubem

%%% jak stworzono backend deploy z githubem i docekrem
%%% ze github oferuje docker-repo

\section{Dodatkowe elementy aplikacji}

\subsection{Tryb mobilny}

\subsection{Motywy aplikacji}

\subsection{Grafiki wygenerowane przez AI}


\chapter{\ChapterTitleWorkOrganization}
\label{sec:organizacja-pracy}

W~tym rozdziale omówione zostały początkowe idee związane z~formą i~funkcjami
aplikacji oraz zarysem harmonogramu działań. Ponadto, przedstawiane
są narzędzia, które zespół wykorzystał do skutecznego zarządzania
i~realizacji projektu. Opisana jest również struktura organizacyjna
zespołu, jego role i~zadania w kontekście tworzenia aplikacji.

Ponadto, zawiera informacje dotyczące przygotowania narzędzi do
weryfikacji postępu pracy, z~myślą o~osobach nadzorujących proces
realizacji projektu.

\section{Charakterystyka i sposób realizacji projektu}

%%% bardzo ogólna wizja co chcemy mieć
%%% jak podzieliliśmy głowne częsci całej aplikacji
%%% core, backend, front, game-engine, thesis

\section{Weryfikacja postępów w trakcie realizacji}

\subsection{Etapy realizacji projektu}
%%% najpierw base, pozniej lobby, pozniej gra a na koncu gra z ai
%%% scisle powiazane z epicami

\subsection{Spotkania z promotorem}
%%% to opisujmey w obu przypadkach \subsubsection{Microsoft Teams}

\subsection{Pracownia projektowa}
%%% to opisujmey w obu przypadkach \subsubsection{Microsoft Teams}


\section{Wykorzystywane narzędzia do realizacji i organizacji pracy}

Do realizacji projektu zostało wykorzystane wiele narzędzi i~technologii
dostępnych za darmo lub jako produkty open source, czyli takie,
których właściciel praw autorskich przyznaje użytkownikom prawa
do swobodnego używania, modyfikacji i~udostępniania oprogramowania.


\subsection{Planowanie i projektowanie}

Przed rozpoczęciem tworzenia pierwszego kodu aplikacji niezbędne było
opracowanie ogólnego planu i~pierwotnej wizji projektu. Zdecydowano
się na przygotowaniu strategii działania oraz
zarysu celów i~funkcji, które mają być zrealizowane w~ramach projektu.
To etap planowania stanowił fundament dla późniejszej implementacji
i~umożliwił efektywne kierowanie procesem tworzenia aplikacji.

Co więcej, w~ramach tego wstępnego etapu, zwrócono uwagę na
opracowanie wizualizacji aplikacji. Starano się osiągnąć jasny
zarys wyglądu aplikacji webowej poprzez stworzenie wstępnych projektów
interfejsu użytkownika. Ta wizualizacja odegrała kluczową rolę
w~procesie projektowania, umożliwiając zespołowi uzyskanie wyobrażenia
o~finalnym wyglądzie aplikacji i~jednocześnie dostosowywanie projektu
do oczekiwań i~potrzeb użytkowników, jak i~wzorowaniu się na utworzonej
wizualizacji.


\subsubsection{Shortcut}

Praktycznie zawsze w~trakcie tworzenia oprogramowania wymagane jest
narzędzie lub platforma udostępniająca możliwość tworzenia zadań,
zależności między nimi oraz dzielenia na większe zbiory zadań.

\begin{figure}[h!]
    \centering
    \includegraphics[width=0.9\textwidth]{img/shortcut/shortcut_backlog.png}
    \caption{Backlog z zadaniami}
\end{figure}

\begin{figure}[h!]
    \centering
    \includegraphics[width=0.9\textwidth]{img/shortcut/shortcut_epic.png}
    \caption{Epic dotyczący systemu lobby wraz z~zadaniami i~ich statusami}
\end{figure}

Zdecydowano się na narzędzie Shortcut \cite{Shortcut}, które oferuje
wszystkie potrzebne funkcjonalności. Pozwala na podział zadań na tzw.
epici -- elementy pracy zawierające wystarczająco dużo zadań do realizacji,
których nie da się ukończyć w~krótkim czasie. Dla każdego milestone'a
utworzono odpowiedni epic, dzięki czemu możliwe było prezentowanie
kolejnych części i~kluczowych etapów procesu tworzenia aplikacji.

\FloatBarrier

Również dużym plusem tego rozwiązania była integracja
z~platformą GitHub\cite{Github}, która pozwoliła na
szybkie wyszukiwanie zmian w~kodzie projektu. Poprzez
zastosowane unikalnych identyfikatorów każde zadanie
znajdujące się w~Shortcut mogło mieć odpowiednio przypisaną
zmianę, która zaszła w~projekcie.


\subsubsection{Figma}

Do utworzenia wstępnej wizualizacji aplikacji użyto narzędzie Figma \cite{Figma}.
Służy ono do projektowania graficznego, takich prac jak szkielety stron
internetowych, prototypy interfejsów użytkownika czy też otwartych przestrzeni
do zarządzania i~planowania.

\begin{figure}[h!]
    \centering
    \includegraphics[width=0.9\textwidth]{img/schematy/milestones.png}
    \caption{Szczegółowy podział milestone'ów na zadania}
\end{figure}

Warto wspomnieć, że tak jak w~\ref{fig:figma_login} lub
\ref{fig:figma_userflow} wykorzystano Figmę do stworzenia mocków i~schematów.

\FloatBarrier

\subsection{Komunikacja wewnętrzna}

W~trakcie tworzenia projektu potrzebna była szybka i~efektywna
wymiana wiadomości oraz możliwość współpracy w czasie rzeczywistym.
Wykorzystana została w~tym celu platforma Discord~\cite{Discord},
która oferuje rozmowy głosowe, kanały tekstowe i~udostępnianie
obrazu na żywo.


\subsection{System kontroli wersji}

Aby synchronizować postępy pracy i~ich historię zastosowano
system kontroli wersji Git~\cite{Git}. Wraz z~nim, do przechowywania
repozytoriów z kodem aplikacji, jak i~dokumentacji wykorzystano
platformę GitHub \cite{Github}.


\subsection{Tworzenie oprogramowania}

Rozwój oprogramowania stanowił kluczowy i~niezbędny element realizacji
założeń projektowych.
Celem było stworzenie funkcjonalnej aplikacji, która integruje
odmienne komponenty, wykorzystujące różne technologie.
W~tym celu skorzystano z~różnych narzędzi, które specjalizowały się
w~tworzeniu i~testowaniu oprogramowania, w~wymaganych przez projekt
technologiach.


\subsubsection{VSCode}

Visual Studio Code \cite{VSCode}, dalej opisywany jako VSCode,
to zaawansowany edytor kodu stworzony
przez Microsoft, który stał się jednym z~najpopularniejszych narzędzi
wśród programistów\cite{IDEIndex}.
VSCode wspiera wiele języków programowania i~znaczników od JavaScript,
TypeScript, Python, C++, Rust po HTML, CSS, JSON i~wiele innych.
Dużą zaletą VSCode jest również możliwość personalizacji. Dostępne są
tysiące rozszerzeń, które pozwalają na dostosowanie edytora do własnych
potrzeb. Użycie tego edytora znacznie usprawniło rozwój aplikacji.

Ze względu na możliwość dostosowania go do różnych języków programowania
i~technologii, stosowano go przy tworzeniu każdego elementu projektu.

Oprócz tworzenia samego oprogramowania w~tym narzędziu wykorzystano go także
do stworzenia dokumentacji aplikacji wraz z~użyciem systemu kontroli wersji Git.
Usprawniło to przebieg pracy nad dokumentem, jak i~wzajemną kontrolę
współtworzenia tekstu.


\subsubsection{Produkty Jetbrains}

Produkty JetBrains \cite{JetBrains}, w~tym dedykowane środowiska
programistyczne (IDE) dla poszczególnych języków i~ekosystemów,
takich jak Java, Python czy PHP.
Oferują one zaawansowane funkcje autouzupełniania i~głębokiej
analizy kodu. Dostarczają one również narzędzia do refaktoryzacji,
co znacząco ułatwia zarządzanie i~optymalizację projektu.

PyCharm \cite{PyCharm}, specjalizujący się w~Pythonie, był pomocny podczas tworzenia
modułu logiki gry w~brydża, a~także w~początkowej fazie konstruowania
serwera z~użyciem FastAPI.

Z~kolei IntelliJ IDEA \cite{Intellij}, ze wsparciem dla języka
Scala, znalazł zastosowanie w~procesie rozwijania ostatecznej wersji
architektury serwerowej.


\subsubsection{Postman}

Postman \cite{Postman} to narzędzie do testowania API, które umożliwia, głównie
programistom i testerom, skuteczne zarządzanie, tworzenie, udostępnianie
i~testowanie żądań HTTP. Jest to popularne środowisko do tworzenia
i~wykonywania zapytań HTTP, zwłaszcza w~kontekście testowania
endpointów RESTful API.

Postman oferuje intuicyjny interfejs graficzny, który ułatwia tworzenie,
wysyłanie i~analizowanie komunikatów sieciowych, a~także sprawdzanie
odpowiedzi serwera. Przydaje się w~wielu etapach procesu tworzenia
aplikacji, od projektowania, wczesnego testowania i~monitorowania API
w~środowisku produkcyjnym.

Przez cały proces tworzenia serwera dla aplikacji Postman był wykorzystywany
do wspomnianych wyżej celów. Było to narzędzie kluczowe w~zapewnieniu
sprawnego działania tej części systemu.


\subsection{Integracja i testowanie}

Podział elementów systemu na wiele części
takich jak aplikacja webowa, serwer, asystent AI
i~dokumentacja wymagał stworzenia odpowiedniego
przepływu pracy zwanego \textbf{CI/CD}. Akronim ten oznacza
ciągłą integrację/ciągłe wdrażanie. Jest on stosowany
w~celu szybkiego likwidowania błędów, przeprowadzania
testów, analizą działania i~integracją elementów systemu
aplikacji. Również pozwala on przyrostowe implementowanie
projektu i~możliwość otrzymania od klienta zwrotnych
informacji o~działaniu systemu.

Zastosowanie tego rozwiązania pozwoliło na
prezentowanie klientowi kolejnych etapów aplikacji, jak
również pozwoliło na niezależne rozwijanie
i~testowanie osobnych części systemu.

Poniżej opisano technologie i~zaprojektowane systemy,
które pozwoliły na skonstruowanie systemu CI/CD dla
tworzonego projektu.


\subsubsection{Github Actions}

Najważniejszym elementem całego systemu było zintegrowanie
repozytorium kodu
wraz z~resztą technologii. Platforma GitHub, na której
znajduje się kod tworzonej aplikacji, oferuje swoje
rozwiązanie GitHub Actions\cite{GithubActions} pozwalające
na zaawansowaną realizację CI/CD. Dla każdego elementu
projektu zostały utworzone instrukcje w~postaci plików JSON
lub wykorzystano integracje zewnętrznych narzędzi oferujące
automatyczne konfiguracje. Instrukcje te były uruchamiane
po aktualizacji kodu.


\subsubsection{Vercel}

Platforma Vercel\cite{Vercel} była odpowiedzialna za
udostępnianie aplikacji webowej. Posiada ona wsparcie \mbox{z~GitHubem},
przez co tworzenie instrukcji odbywa się automatycznie.
Vercel, dla głównej gałęzi kodu \textit{"master"} oraz
gałęzi implementujących dodatkowe funkcje, budował aplikację
i~udostępniał na swojej platformie. W~przypadku aktualizacji
kodu, na odpowiednich gałęziach, przystępował on do ponownego
budowania aplikacji i~zastępował starą. Dla każdej gałęzi
generowany był osobny adres, więc w~przypadku zmian zawsze
najnowsza wersja aplikacji, z~danej gałęzi, była dostępna pod
tym samem odpowiadającym jej adresem. Co więcej, w przypadku
problemów lub błędów podczas budowania aplikacji, informuje
on o~tym i~udostępnienia zapisy logów, aby można było je
zweryfikować.

\begin{figure}[hbt!]
    \centering
    \includegraphics[width=0.9\textwidth]{img/github/github-vercel.png}
    \caption{Wycinek integracji platformy Vercel z~GitHub dla jednej z~gałęzi kodu aplikacji webowej}
    \label{fig:github-vercel}
\end{figure}

Na rysunku
\ref{fig:github-vercel} można zauważyć, jak dla jednej
z~gałęzi został wygenerowany adres do aplikacji
("Visit Preview") oraz że budowanie przebiegło bez problemów
(zielone zaznaczenie).

\FloatBarrier


\subsubsection{Oracle Cloud}

Do uruchomienia serwera naszego systemu wymagane było
osobne środowisko. Wymagane było, aby oferowało ono
narzędzie do automatycznego budowania i~udostępniania
najnowszych wersji serwera, w~podobny sposób jak zostało
to rozwiązane w~przypadku aplikacji webowej.

Rozwiązano ten problem poprzez stworzenie autorskiego
narzędzia wraz z~wykorzystaniem serwera udostępnionego
przez platformę Oracle Cloud \cite{OracleCloud}.
Wykorzystując możliwość tworzenia instrukcji GitHub Actions
skonfigurowano automatyczne tworzenie obrazu kontenera
Docker \cite{Docker}, który miał za zadanie uruchamiać
serwer. Był on udostępniany w~rejestrze kontenerów na
platformie GitHub. Następnie po stronie serwera Oracle
cyklicznie był sprawdzany rejestr w~celu wykrycia
nowo powstałego obrazu kontenera. Gdy obraz posiadał
inny hash\footnote{hash -- wartość wyliczana przez tzw. "funkcję
    hashującą", która dla otrzymanych na wejściu tych
    samych danych zwraca tę samą wartość.}
odpowiedni skrypt uruchamiał on nową wersję kontenera.
Każda gałąź kodu, podobnie jak w~przypadku aplikacji
webowej, posiadała osobny kontener uruchomiony na serwerze
Oracle, do którego tworzony był osobny adres za pomocą
autorskich skryptów i~oprogramowania Traefik \cite{Traefik}.
Dzięki temu aplikację webową można było połączyć
z~dowolnie wybraną wersją serwera gry, na przykład, aby
przetestować nową funkcjonalność. Na rysunku
\ref{fig:traefik-dashboard} przedstawiono panel, na którym
znajduje się lista serwerów udostępnionych dla
odpowiadających im gałęzi kodu.

\begin{figure}[hbt!]
    \centering
    \includegraphics[width=0.9\textwidth]{img/traefik/dashboard.png}
    \caption{Panel serwisów udostępnionych przez Traefik.}
    \label{fig:traefik-dashboard}
\end{figure}


\subsection{Tworzenie dokumentacji}

Do napisania dokumentacji aplikacji użyto języka \LaTeX~\cite{Latex}.
Przesyłając tekst pracy na platformę GitHub, podobnie skorzystano
\mbox{z GitHub Actions}, którą wykorzystano w celu
generacji dokumentu natychmiast po opublikowaniu zmian w~tekście pracy.
Umożliwiło to na dostęp do najnowszej wersji dokumentu dla opiekuna
pracy i~prowadzącego pracownię projektową, dzięki czemu była możliwa
ciągła weryfikacja postępów nad pracą.



\chapter{\ChapterTitleResults}
\label{sec:wyniki-projektu}

\section{Weryfikacja realizacji wstępnych założeń}

W~pierwszym etapie planowania projektu utworzony został ogólny
plan, który opisywał oddzielne systemy i~elementy
zleconej aplikacji (Rys. \ref{fig:figma_strategicplan}).
Uwzględnia on nie tylko wymagania zdefiniowane w~rozdziale drugim
\nameref{sec:zakres-funkcjonalnosci}, ale także potrzebne do ich
realizacji wymagania techniczne.

\begin{figure}[h!]
  \centering
  \includegraphics[width=0.9\textwidth]{img/schematy/milestones.png}
  \caption{Szczegółowy podział milestone'ów na zadania}
  \label{fig:figma_strategicplan}
\end{figure}

Opierając się na powyższym planie, można stwierdzić, że udało się
zrealizować wszystkie wymagania sprecyzowane przez klienta.
Jedynym elementem, który został zrealizowany alternatywnie,
zgodnie z~założeniami zagrożeń implementacji
(\nameref{sec:analiza_zagrozen}), był wirtualny asystent. Szczegóły
dotyczące tej części projektu zostały opisane w~sekcji dotyczącej
asystenta \nameref{subsubsec:mocai}.

\section{Przegląd zrealizowanych funkcjonalności}

Poniższa sekcja opisuje wyniki zrealizowanego przez zespół
projektu. Przedstawiona jest aplikacja oraz wyniki, jakie
osiąga wirtualny asystent.

\subsection{Interfejs aplikacji}

Interfejs aplikacji był tworzony z~uwagą na postawione wymagania funkcjonalne.
Zapewnia on użytkownikowi dostęp do wszystkich funkcji niezbędnych do korzystania
z~aplikacji. Widoki aplikacji są w~znacznym stopniu podobne do szkiców wykonanych na etapie
rozpoczęcia prac. Różnią się nieznacznie stylem ze względu na użycie innych narzędzi
do ich utworzenia. Są też bardziej szczegółowe od szkiców, ponieważ nie planowano
rozmieszczenia najdrobniejszych elementów interfejsu, takich jak przyciski akcji hosta
w~lobby czy oznaczenia pozycji graczy na ekranie gry. Przykładowe porównania końcowych
rezultatów ze szkicami są przedstawione na rysunkach \ref{fig:compare_lobby}
i~\ref{fig:compare_game}.

\begin{figure}[h!]
  \centering
  \subfloat[Ekran lobby]{
    \includegraphics[width=0.45\textwidth]{img/widoki/lobby.png}
  }%
  \hspace*{0.5cm}
  \subfloat[Szkic lobby]{
    \includegraphics[width=0.45\textwidth]{img/figma-szkic/3.png}
  }%
  \caption{Porównanie ekranu lobby z początkowym szkicem}
  \label{fig:compare_lobby}
\end{figure}

\begin{figure}[h!]
  \centering
  \subfloat[Ekran gry]{
    \includegraphics[width=0.45\textwidth]{img/widoki/game.png}
  }%
  \hspace*{0.5cm}
  \subfloat[Szkic gry]{
    \includegraphics[width=0.45\textwidth]{img/figma-szkic/4.png}
  }%
  \caption{Porównanie ekranu gry z początkowym szkicem}
  \label{fig:compare_game}
\end{figure}

\FloatBarrier

\subsubsection{Tworzenie i dołączenie do gry}

Na stronie głównej aplikacji (Rys. \ref{fig:home}) użytkownik może utworzyć nową
rozgrywkę poprzez naciśnięcie przycisku "Create a game" lub dołączyć do istniejącej
poprzez wpisanie kodu gry otrzymanego od innego użytkownika i~naciśnięcie
przycisku "Join".

\begin{figure}[h!]
  \centering
  \includegraphics[width=0.9\textwidth]{img/widoki/home.png}
  \caption{Strona główna aplikacji}
  \label{fig:home}
\end{figure}

\FloatBarrier

\subsubsection{Lobby}

Strona lobby (Rys. \ref{fig:lobby}) służy do zebrania
użytkowników przed rozpoczęciem gry i~ustawienia ich
na odpowiednich pozycjach. Widok zawiera cztery pozycje
z~pseudonimami graczy, którzy się na nich znajdują.
Host jest oznaczony ikoną korony, znajdującą się nad nazwą
jego pozycji.

\begin{figure}[h!]
  \centering
  \includegraphics[width=0.9\textwidth]{img/widoki/lobby.png}
  \caption{Strona lobby}
  \label{fig:lobby}
\end{figure}

\FloatBarrier

\subsubsection{Host}

Zgodnie z~założeniami, host ma możliwość wyrzucania graczy
z~lobby, zmieniania ich pozycji oraz obsadzania pustych
miejsc graczami-asystentami. Ponadto, host może oddać swoje
uprawnienia innemu graczowi-użytkownikowi. Te akcje może
wykonywać poprzez użycie odpowiednich przycisków w~postaci
ikon znajdujących się nad pozycjami graczy (Rys. \ref{fig:host_actions_ui}).

\subsubsection{Użytkownik}

Użytkownik ma możliwość w~dowolnym momencie opuścić lobby, naciskając przycisk "Leave",
oraz zgłosić gotowość do rozpoczęcia rozgrywki, poprzez naciśnięcie przycisku "Ready".
Dodatkowo może skopiować kod gry za pomocą przycisku "Copy ID" i~przesłać go innemu
użytkownikowi. Warto zaznaczyć, że kod nie jest widoczny w~żadnym momencie, co może być
przydatne dla graczy udostępniających ekran podczas rozgrywki, aby zabezpieczyć się przed
nieproszonymi uczestnikami.

\subsection{Gra w brydża}

Utworzony został prosty interfejs gry, który zapewnia
wymagane elementy gry jak system licytacji i~widoczne
karty odpowiednich graczy. Ograniczono się głównie do
zaimplementowania tych akcji, jakie mógłby wykonać także
wirtualny asystent.

\subsubsection{Licytacja}

Podczas licytacji użytkownik może na bieżąco obserwować
podejmowane kontrakty przez innych graczy. Niedostępne akcje
są blokowane, co ułatwia wybór odpowiedniego kontraktu.
Wskazywany jest również gracz, który w~danym momencie
mógłby wygrać licytację, gdy trzech kolejnych graczy
pasuje. Przykładowy stan licytacji widoczny jest na rysunku
\ref{fig:bidding}.

\begin{figure}[h!]
  \centering
  \includegraphics[width=0.9\textwidth]{img/widoki/bidding.png}
  \caption{Interfejs gry podczas licytacji}
  \label{fig:bidding}
\end{figure}

\FloatBarrier

\subsubsection{Rozgrywka}

W~ekranie rozgrywki (Rys. \ref{fig:game}) znalazły się takie
elementy jak widok
kart każdego z~graczy, w~tym widoczne są awersy kart
użytkownika i~dziadka (lub partnera, gdy użytkownik jest
dziadkiem). Litery znajdujące się przy kartach określają
kierunek, na którym się znajdują. Podświetlany jest
kierunek tego gracza, który aktualnie wykonuje ruch.
Na środku ekranu znajdują się zagrane karty tworzące lewę.
Wszystkie karty i~ich zagrania są płynnie animowane,
dzięki czemu użytkownicy mogą doświadczyć gry
w~podobny sposób jak przy prawdziwym stole.

\begin{figure}[h!]
  \centering
  \includegraphics[width=0.9\textwidth]{img/widoki/game.png}
  \caption{Interfejs gry podczas rozgrywki}
  \label{fig:game}
\end{figure}

\FloatBarrier

\subsection{Wirtualny asystent}
%%% przeprowadzone testy aplikacji na wybranej grupie osób (do wywalenia?)

W fazie planowania projektu wskazano 5 różnych algorytmów AI, które mogłyby zostać
wykorzystane.
Nie podjęto próby implementacji pierwszego z nich (Regularized Nash Dynamics),
ze względu na jego złożoność i~brak dostępnych implementacji.
Zdecydowano się na implementację drugiego algorytmu (AlphaZero).
Początkowo planowano wykorzystanie rozszerzenia IS-MCTS.
Odkryto jednak sposób na zaimplementowanie algorytmu bez tego rozszerzenia,
co pozwoliło na uproszczenie implementacji. Nazywamy tę wersję algorytmu
\textit{Depth-Zero AlphaZero}.


\subsubsection{Poziom sztucznej inteligencji}
\label{subsubsec:mocai}

Agent AlphaZero uczony był poprzez granie sam ze sobą.
W~procesie uczenia zaobserwował ponad 110~mln ruchów,
co odpowiada około 10 mln gier, zakładając średnią
długość licytacji 11 ruchów.
W~celach ewaluacji agenta, przeprowadzane
były gry pomiędzy agentem a~graczem wykonującym
losowe ruchy.
Nie jest to najlepszy przeciwnik, ale pozwala
ocenić, czy agent polepsza swoje umiejętności.
Ewaluacja wykorzystywała samą sieć $\pi$,
agent nie widział kart innych graczy.
Wyniki przedstawia Rys. \ref{fig:bzero-eval}.
Agent szybko osiąga 95\% wygranych.
Brak 100\% wygranych wynika z~faktu, że
podczas ewaluacji agent nie wybiera zawsze optymalnego ruchu,
tylko losuje ruch z~rozkładu prawdopodobieństwa
wygenerowanego przez $\pi$.
Służy to zróżnicowaniu gier i~umożliwia
dokładniejszą ocenę umiejętności agenta.


\begin{figure}[!]
    \centering
    \includegraphics[width=\textwidth]{img/wykresy/bzero-eval.png}
    \caption{Proces uczenia AlphaZero vs. losowy gracz -- Licytacja}
    \label{fig:bzero-eval}
\end{figure}

Ze względu na brak odpowiedniego modelu
bazowego (baseline) przeciwnika,
przeprowadzono test zaimplementowanego algorytmu
na grze Othello \cite{Othello}.
Uruchomiono identyczny proces uczenia,
wykorzystując implementację Othello oraz
model AI baseline z~\cite{PGX}.
Wyniki uczenia przedstawia Rys.~\ref{fig:othello-eval}.
Nasze AI uzyskuje stosunek wygranych do przegranych
kontra model AI baseline na poziomie 9:1.
Można zatem wnioskować, że nasza implementacja
AlphaZero jest poprawna.


\begin{figure}[h!]
  \centering
  \includegraphics[width=\textwidth]{img/wykresy/othello-eval.png}
  \caption{Proces uczenia AlphaZero vs. PGX Baseline -- Othello}
  \label{fig:othello-eval}
\end{figure}

AI do fazy rozgrywki wykorzystuje algorytm DDS,
zatem jest matematycznie doskonałe i~nie ma potrzeby
oceny jego umiejętności.


\subsubsection{Możliwości rozwoju modelu}
%%% czy da sie rozszerzac o kolejne elementy gry brydza
%%% mozna sie odwolac do dalszych perspektyw (na ten moment w PR)

Efektem implementacji AlphaZero było stworzenie
rozbudowanego środowiska do uczenia.
W~przyszłości można wykorzystać to środowisko
do uczenia innych algorytmów AI, między innymi
Regularized Nash Dynamics.
Znaczna część kodu jest wspólna dla obu algorytmów.
W~przyszłości można również rozszerzyć
model o~dodatkowe elementy gry w~brydża,
na przykład o~systemy licytacyjne.



\subsection{Ułatwienia dostępności}

Oprócz omówionych kluczowych funkcjonalności, w~ramach rozwoju
aplikacji wprowadzono także dodatkowe elementy, których
potrzebę przedstawiono w~podrozdziale dotyczącym wymagań
niefunkcjonalnych. Głównych ich celem było zapewnienie
wygody i~prostego korzystania z~aplikacji, nawet podczas
pierwszego użytkowania. Poniżej przedstawione rozwiązania powstały
z~inicjatywy zespołu. Zostały one zaakceptowane przez klienta,
realizując wymagania dostępności i~użyteczności.

\subsubsection{Intuicyjność interfejsu}

Aby aplikacja była intuicyjna dla każdego użytkownika,
zastosowano ikony wprost kojarzące się z~wykonywaną
przez nie akcją. Zastosowano wyróżniające się kolory wśród palety
aplikacji w~celu podkreślenia danych czynności wykonywanych
przez użytkownika (Rys. \ref{fig:host_actions_ui}). Kontrastowe barwy mają zwrócić uwagę na
skutki danej akcji. Przykładowo jaskrawy czerwony kolor
kojarzący się przemocą lub impulsywnością wraz z~ikoną
przedstawiającą "$\times$"\xspace odpowiada za wyrzucenie gracza
z~sesji. Zastosowanie wspomnianych technik przestrzega
użytkownika przed przedwczesnym kliknięciem, ale także kojarzą
się one z~negatywną czynnością.

\begin{figure}[h!]
  \centering
  \includegraphics[width=0.9\textwidth]{img/widoki/host_actions.png}
  \caption{Akcje hosta lobby}
  \label{fig:host_actions_ui}
\end{figure}

\FloatBarrier

\subsubsection{Responsywny układ aplikacji}

Zgodnie z~wymogiem dostępności interfejsy aplikacji powinny
być funkcjonalne również na urządzeniach o~niewielkich
rozmiarach ekranu. Wszystkie strony zostały zaprojektowane
tak, aby umożliwić wygodne korzystanie zarówno na urządzeniach
stacjonarnych, jak i~mobilnych (Rys. \ref{fig:responsive_ui}). Aplikacja dynamicznie
dostosowuje się w~zależności od aktualnego rozmiaru okna
przeglądarki.

Minimalna przewidziana
szerokość ekranu wynosi 280 pikseli, dzięki czemu wspierane
są także starsze urządzenia o~niewielkiej rozdzielczości
ekranu.

\begin{figure}[h!]
  \centering
  \includegraphics[width=0.9\textwidth]{img/widoki/desktop_mobile.png}
  \caption{Tryb mobilny i desktopowy aplikacji}
  \label{fig:responsive_ui}
\end{figure}

\FloatBarrier

\subsubsection{Motywy jasny i ciemny}

Wygląd aplikacji został zrealizowany w~dwóch trybach --
jasnym i~ciemnym (Rys. \ref{fig:themes_ui}). Użyto w~tym celu dostępnych w~bibliotece
palet kolorów, ale także zdefiniowano własne, aby zachować
motyw kolorystyczny aplikacji. W~zależności od aktualnie
wybranego motywu kolory zmieniały swój odcień.

% \begin{figure}[h!]
%   \centering
%   \includegraphics[width=\textwidth, draft=true]{example-image}
%   \caption{Motywy jasny i ciemny aplikacji}
%   \label{fig:themes_ui}
% \end{figure}

% \FloatBarrier

\begin{figure}[h!]
  \centering
  \begin{minipage}[b]{0.45\textwidth}
    \centering
    \includegraphics[width=\textwidth]{img/widoki/light.png}
  \end{minipage}%
  \hspace*{0.5cm}
  \begin{minipage}[b]{0.45\textwidth}
    \centering
    \includegraphics[width=\textwidth]{img/widoki/dark.png}
  \end{minipage}
  \caption{Motywy jasny i ciemny aplikacji}
  \label{fig:themes_ui}
\end{figure}

\FloatBarrier

\section{Dalsze perspektywy rozwoju projektu}

Aplikacja do gry w~brydża została stworzona głównie
na potrzeby asystenta, aby można było go wykorzystać w~grze
poprzez sieć Internet. Tworzona była w~technologiach, które
w~naszej opinii są popularne, a~ich rozwój jest stale wspierany.
Dzięki temu możliwy jest przyszły rozwój lub przebudowa aktualnej
wersji aplikacji.
W~związku z~tym zaproponowano potencjalne funkcjonalności rozszerzające
możliwości asystenta i~aplikacji, które
mogłyby być rozwinięte w~przyszłości. \\

Jedną z~ważniejszych funkcji jest wprowadzenie wsparcia do
przeprowadzenia pełnej rozgrywki gry. Na ten moment możliwe jest
rozegranie jednego rozdania. Przy wprowadzeniu pełnej gry należałoby
utworzyć zapisy wyników z~każdej partii, w~których skład wchodzą rozdania,
a~także dostosować asystenta
do uwzględniania wyników gry\footnote{Uwzględnienie wyników gry jest
  rozumiane jako dodanie wyników gry do obserwacji modelu asystenta.}
podczas podejmowania decyzji. Istotne w~tym przypadku
byłoby również rozszerzenie interfejsu gry, aby mógł zawierać
dodatkowe informacje, jak na przykład aktualny wynik gry lub
historię licytacji.

Również istotną propozycją, która z~pewnością podniesie poziom
współpracy asystenta z~graczem, jest wsparcie języków licytacji\footnote{
  język licytacji -- odzywki licytacyjne mające przekazać informacje
  partnerowi, którym przypisane
  są ustalone informacje o~posiadanych kartach.
} (systemów licytacji).
Aktualnie asystent podejmuje decyzje podczas licytacji bez znajomości
języków. Oznacza to, że nie rozumie odzywek gracza w~stosowanym przez
niego systemie licytacyjnym.

Dodatkową funkcjonalnością, która rozszerza zastosowanie asystenta, jest
wykorzystanie go do analizy rozgrywek w~czasie rzeczywistym lub
ich historycznych zapisów. Dzięki temu gracz mógłby uzyskać podpowiedzi
o~możliwie najlepszych ruchach na podstawie aktualnego stanu gry.
W~przypadku gier już odbytych mógłby także sprawdzić, w~jakich etapach
zostały popełnione błędy lub wykonano najlepsze z~możliwych ruchów.

Również można rozwinąć analizę przeprowadzoną przez asystenta, aby
podpowiedzi do gry były zintegrowane z~modelem językowym. Dzięki
takiemu rozwiązaniu gracz mógłby korzystać z~chatu dostępnego
z~poziomu aplikacji, aby zadać pytanie o~aktualnie przeprowadzanej
rozgrywce lub na temat zasad gry w~brydża. Model językowy zapewniłby
zrozumiałe dla gracza odpowiedzi i~w~razie potrzeby mógłby bardziej
szczegółowo sprecyzować odpowiedź na prośbę gracza.

\section{Podsumowanie}

Uczestnictwo przez zespół w~tworzeniu projektu
było wielkim wyzwaniem dla każdego z~jego członków.
Doświadczyliśmy wielu problemów, z~którymi nie
zmierzyliśmy się wcześniej. Podczas tworzenia
produktu wykorzystaliśmy własne umiejętności nabyte podczas
toku studiów, jak i~sami poznaliśmy wiele nowych technologii.
Odkryliśmy wiele rozwiązań i~sposobów na realizacje
platformy webowej i~udało nam się wdrożyć je do naszej aplikacji.
Rozszerzyliśmy wiedzę z~zakresu połączeń sieciowych, znajdując
rozwiązanie oferujące niskie opóźnienie pomiędzy serwerem gry
a~klientem. Po raz pierwszy zetknęliśmy się z~tworzeniem
środowiska z~zastosowaniem trójwymiarowej grafiki komputerowej
oraz przebadaliśmy wiele informacji dotyczących sztucznej
inteligencji i~możliwości zastosowania jej do tak złożonej gry
jak brydż.

Poprzez analizę
wymagań, planowanie, analizę zagrożeń, implementację
i~testowanie utworzono pełny system spełniający wymagania
klienta.
Stworzona aplikacja jest w~pełni użyteczna i~może być
wykorzystywana w~sposób darmowy do grania w~brydża przez Internet.

Jesteśmy usatysfakcjonowani z~finalnego rozwiązania produktu,
które dostarczyliśmy i~zadowoleni, że udało nam się pokonać
większość trudności podczas jego realizacji.





%%%%%%%%%%%%%%%%%%%%%%%%%%%%%%%%%%%%%%%%%%%%%%%%%%%%%%%%%%%%%%%%%%%%%%%%%%%%%%%

\printbibliography

%%%%%%%%%%%%%%%%%%%%%%%%%%%%%%%%%%%%%%%%%%%%%%%%%%%%%%%%%%%%%%%%%%%%%%%%%%%%%%%

\listoffigures
\listoftables
\listofalgorithmes
\lstlistoflistings

\end{document}
